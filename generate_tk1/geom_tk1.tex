\documentclass[a4paper, 12pt]{article}
\usepackage{cmap}					% поиск в PDF
\usepackage{mathtext} 				% русские буквы в формулах
\usepackage[T2A]{fontenc}			% кодировка
\usepackage[utf8]{inputenc}			% кодировка исходного текста
\usepackage[english,russian]{babel}	% локализация и переносы
\usepackage{indentfirst}
\frenchspacing
\usepackage{geometry}
\geometry{left=3cm, right=1.5cm, top=2cm, bottom=1.5cm}
\usepackage{ifthen}
\usepackage{sagetex}
\usepackage{soul}
\newcounter{zno} % счетчик номеров задач
%%% Дополнительная работа с математикой
\usepackage{amsmath,amsfonts,amssymb,amsthm,mathtools,bm} %
\usepackage{icomma} % "Умная" запятая: $0,2$ --- число, $0, 2$ --- перечисление
\usepackage{etoolbox}
\usepackage{enumitem}

\usepackage{lastpage}
\usepackage{fancyhdr} % Колонтитулы
\pagestyle{fancy}
\renewcommand{\headrulewidth}{0.5pt}  % Толщина линейки, отчеркивающей верхний колонтитул
\renewcommand{\footrulewidth}{0.5pt}
\lfoot{\textit{В.~Н. Кожухова}}
\rfoot{%Иванов Иван Иванович
	\input{\jobname.txt}%
}
%\rhead{Верхний правый}
\chead{\textbf{ДИТИ НИЯУ МИФИ}, осенний семестр 2017-2018 уч. года}
%\lhead{Верхний левый}
%\cfoot{c. \thepage \; из \pageref{LastPage}} % По умолчанию здесь номер страницы


\makeatletter
\AddEnumerateCounter{\asbuk}{\russian@alph}{щ}
\makeatother
%\usepackage[explicit]{titlesec}

%\renewcommand{\thesection}{Текущий контроль \arabic{section}.}
%\newbool{answers}
%\booltrue{answers}
%\newcommand{\zadacha}[2][Нет ответа.]{%

%\addtocounter{zno}{1}
%Задача \arabic{zno}. #2%

%\ifbool{answers}{\textbf{Ответ}. #1 \vspace{5mm}}{\vspace{5mm}}
%}
%\newcommand{\zadacha}[2]{%
%\refstepcounter{zno}\label{#1}
%\textbf{Задача \arabic{zno}}. #2%
%}

%\newcommand{\mycomment}[1]{\hphantom{#1}}
%\newcommand{\mycomment}[1]{#1}

%\def\teachers{}
%\newcommand{\mycomment}[1]{\ifdef{\teachers}{#1}{\hphantom{#1}}}	

\usepackage{exsheets}
\DeclareTranslation{Russian}{exsheets-exercise-name}{Задача}
\DeclareTranslation{Russian}{exsheets-question-name}{Вопрос}
\DeclareTranslation{Russian}{exsheets-solution-name}{Ответ}
\SetupExSheets[points]{name=б.}

%\def\teachers{}
\newcommand{\mycomment}{\ifdef{\teachers}{\printsolutions}{}}

\begin{document}
\thispagestyle{empty}
\vspace{-10ex}

\begin{center}
	\Huge{Аналитическая геометрия}\\
	\vspace{-3ex}
	\section*{Текущий контроль 1}
\end{center}
\vspace{-2ex}	
\textit{Стоимость: \pointssum* б.}
\begin{flushleft}
	Направление подготовки: 09.03.02 Информационные системы и технологии %14.03.02 Ядерные физика и технологии  03.03.02 Физика 
	
	ФИО студента: %Иванов Иван Иванович
	\input{\jobname.txt}
	
	Преподаватель: \textit{В.",Н. Кожухова}	
\end{flushleft}	

\subsection*{Векторная алгебра}
\begin{sagesilent}
# 1) Скалярное и векторное произведение векторов через координаты
a1 = vector([randint(-7, 7) for i in range(3)])
b1 = vector([randint(-4, 4) for i in range(3)])
anga1b1 = a1.dot_product(b1)/(a1.norm()*b1.norm())
pra1b1 = a1.dot_product(b1)/b1.norm()
prc=0
while prc==0:
   prc = randint(-6,6)
s1=[]
while s1==[]:
 var('x,y,z')
 ch1 = randint(1,3)
 if ch1 == 1 :
   os = r"Ox"
   eqn = [a1[0]*x+a1[1]*y+a1[2]*z==0, b1[0]*x+b1[1]*y+b1[2]*z==0,x == prc]
 else :
    if ch1 == 2 :
       os = r"Oy"
       eqn = [a1[0]*x+a1[1]*y+a1[2]*z==0, b1[0]*x+b1[1]*y+b1[2]*z==0,y == prc]
    else :
       os = r"Oz"
       eqn = [a1[0]*x+a1[1]*y+a1[2]*z==0, b1[0]*x+b1[1]*y+b1[2]*z==0,z == prc]
 s1 = solve(eqn, x,y,z)
# 2) Разложение вектора по векторам
crossm = zero_matrix(3,3)
while (det(crossm)==0): 
   p = vector([randint(-7, 7) for i in range(3)])
   q = vector([randint(-3, 3) for i in range(3)])
   r = vector([randint(-10, 10) for i in range(3)])
   crossm = matrix([p, q, r])
coord = zero_vector(3)
while coord == zero_vector(3) : 
   coord = vector([randint(-3, 3) for i in range(3)])
x2 = coord[0]*p + coord[1]*q + coord[2]*r
# 3) Площадь параллелограмма со сторонами |u| и |v|
S3 = FiniteEnumeratedSet([30,45,60,90,135,120,150]) # углы в градусах
ang = S3.random_element() # случайный угол
normA = randint(1,7) # длины векторов a и b
normB = randint(1,5)
ucoord = zero_vector(2)
vcoord = zero_vector(2)
while ucoord==zero_vector(2) or vcoord==zero_vector(2) or det(matrix([ucoord,vcoord]))==0:
  au = randint(-5,5) # u = au*a+bu*b -- разложение u по a и b
  bu = randint(-6,6) # генерим случайные коэффициенты разложения
  av = randint(-7,7) # v = av*a+bv*b -- разложение v по a и b
  bv = randint(-4,4)
  ucoord = vector([au,bu])
  vcoord = vector([av,bv])
var('a,b')
basisab = vector([a,b])
u = basisab.dot_product(ucoord)
v = basisab.dot_product(vcoord)
muv = matrix([ucoord, vcoord])
Sparal = abs(normA*normB*sin(ang*pi/180)*det(muv))  
ad1 = au+av # коэф-ты разложения векторов диагоналей: u+v ; u-v
bd1 = bu+bv
ad2 = au-av
bd2 = bu-bv
# скалярное произведение векторов диагоналей
d1d2 = ad1*ad2*normA^2+(ad1*bd2+ad2*bd1)*normA*normB*cos(ang*pi/180)+bd1*bd2*normB^2
# длины диагоналей
normd1 = sqrt(ad1*ad1*normA^2+2*ad1*bd1*normA*normB*cos(ang*pi/180)+bd1*bd1*normB^2)
normd2 = sqrt(ad2*ad2*normA^2+2*ad2*bd2*normA*normB*cos(ang*pi/180)+bd2*bd2*normB^2)
# угол между диагоналями
angd1d2 = d1d2/(normd1*normd2)
# 4) Проверка компланарности векторов
choose = randint(1,2) #1 - ЛЗ, #2 ЛНЗ
if choose==1 :
  a4 = vector([randint(-7, 7) for i in range(3)])
  b4 = vector([randint(-3, 3) for i in range(3)])
  c4 = randint(-5,5)*a4 + randint(-5,5)*b4
else :
  a4 = vector([randint(-7, 7) for i in range(3)])
  b4 = vector([randint(-3, 3) for i in range(3)])
  c4 = vector([randint(-5, 5) for i in range(3)])
abc = c4.dot_product(a4.cross_product(b4))  
if abc == 0 :
    compl = r"компланарны"
else :
    compl = r"не компланарны"
# 5) Вычисление объема тетраэдра, вершины А1, А2, А3, А4
A1 = vector([randint(-7, 7) for i in range(3)])
A2 = vector([randint(-5, 5) for i in range(3)])
A3 = vector([randint(-3, 3) for i in range(3)])
A4 = vector([randint(-5, 5) for i in range(3)])
A1A2 = A2-A1
A1A3 = A3-A1
A1A4 = A4-A1
Sosn = (A1A2.cross_product(A1A3)).norm()/2
Vtetr = abs(A1A4.dot_product(A1A2.cross_product(A1A3)))/6
htetr = 3*Vtetr/Sosn
# 6) Работа сил
f1 = vector([randint(-7, 7) for i in range(3)])
f2 = vector([randint(-6, 6) for i in range(3)])
f3 = vector([randint(-3, 3) for i in range(3)])
var('i,j,k')
mybasis = vector([i,j,k])
A6 = vector([randint(-3, 3) for i in range(3)])
B6 = vector([randint(-5, 5) for i in range(3)])
A6B6 = B6-A6
f = f1 + f2 + f3
Answ6 = f.dot_product(A6B6)
# 7) Моменты сил
C7 = vector([randint(-7, 7) for i in range(3)])
D7 = vector([randint(-5, 5) for i in range(3)])
G7 = vector([randint(-6, 4) for i in range(3)])
D7C7 = C7 - D7
G7C7 = C7 - G7
MD = D7C7.cross_product(f)
MG = G7C7.cross_product(f)
\end{sagesilent}

\vspace{2ex}	

\begin{question}{1}\label{z1}
	Даны векторы $\bm a$ и $\bm b$. 
	\[\bm{a}=\sage{a1}, \bm{b}=\sage{b1}\] 
	Найдите:
	\begin{enumerate}%[label=\asbuk*),ref=\asbuk*]
		\item скалярное и векторное произведение векторов $\bm a$ и $\bm b$; 
		\item угол между векторами $\bm a$ и $\bm b$; 
		\item проекцию вектора $\bm a$ на вектор $\bm b$; 
		\item вектор $\bm c$, перпендикулярный векторам $\bm a$ и $\bm b$, если проекция вектора $\bm c$ на ось \textit{\sagestr{os}} равна $\sage{prc}$. 
	\end{enumerate}
\end{question}	
\begin{solution}
	\ensuremath{
		\displaystyle{\left(\bm{a},\bm{b}\right)=\sage{a1.dot_product(b1)}, \left[\bm{a},\bm{b}\right]=\sage{a1.cross_product(b1)}, \cos\alpha = \sage{anga1b1}, \, pr_{\bm b}\bm a = \sage{pra1b1}}}, \\
	$\bm c = \sage{s1[0]}$  
\end{solution}

\begin{question}{1}
	Разложить вектор $\bm x$ по векторам $\bm p, \bm q, \bm r$ : $\bm x = \alpha\bm p + \beta\bm q + \gamma\bm r$, если
	\[\bm{x}=\sage{x2}, \bm{p}=\sage{p}, \bm{q}=\sage{q}, \bm{r}=\sage{r}\]
\end{question}
\begin{solution}
	\ensuremath{\alpha = \sage{coord[0]}, \beta = \sage{coord[1]}, \gamma = \sage{coord[2]}}
\end{solution}

\begin{question}{1}
	Вычислить площадь параллелограмма, построенного на векторах $\bm u$ и $\bm v$, и найти угол между его диагоналями, если:
	$\bm{u}=\sage{u}, \bm{v}=\sage{v}$, при этом о векторах $\bm a$ и $\bm b$ известно, что: 
	$|\bm a|=\sage{normA}, |\bm b| = \sage{normB}$,
	$\angle\left(\bm a, \bm b\right)=\sage{ang}^\circ$.
\end{question}
\begin{solution}
	\ensuremath{\displaystyle{S = \sage{latex(Sparal)}, \cos\angle\left(\bm{d_1}, \bm{d_2}\right)=\sage{angd1d2}}}
\end{solution}

\begin{question}{1}
	\begin{enumerate}
\item Проверить, компланарны ли векторы $\bm a$, $\bm b$ и $\bm c$:
\[\bm{a}=\sage{a4}, \bm{b}=\sage{b4}, \bm{c}=\sage{c4}\]
\item Вычислить объем тетраэдра с вершинами в точках $A_1, A_2, A_3, A_4$ и его высоту, опущенную из вершины $A_4$ на грань $A_1 A_2 A_3$:
\[A_1\sage{A1}, A_2\sage{A2}, A_3\sage{A3}, A_4\sage{A4}\]
	\end{enumerate}	
\end{question}
\begin{solution}
	Векторы \sagestr{compl}, \ensuremath{\bm{abc} = \sage{abc}}
	
	Объем тетраэдра \ensuremath{\displaystyle{V = \sage{Vtetr}}}, длина высоты \ensuremath{\displaystyle{h = \sage{htetr}}}
\end{solution}

\begin{question}{1}
	На материальную точку действуют три силы: $\bm{f_1}, \bm{f_2} \text{ и } \bm{f_3}$:
	\[\bm{f_1}=\sage{mybasis.dot_product(f1)}, \bm{f_2}=\sage{mybasis.dot_product(f2)}, \bm{f_3}=\sage{mybasis.dot_product(f3)}\]
	 \begin{enumerate}
\item Найти работу равнодействующей этих сил по перемещению данной материальной точки из точки $A$ в точку $B$, если
$A\sage{A6}, B\sage{B6}$.
\item Cилы $\bm{f_1}, \bm{f_2} \text{ и } \bm{f_3}$ приложены к точке $C\sage{C7}$. Сравните модули моментов равнодействующей этих сил относительно точек $D\sage{D7}$ и $G\sage{G7}$.
	 \end{enumerate}
\end{question}
\begin{solution}
	$A=\sage{Answ6}$
	
	$M_D (C)=\sage{MD.norm()}\approx\sage{n(MD.norm(), digits=4)}, M_G (C)=\sage{MG.norm()}\approx\sage{n(MG.norm(), digits=4)}$
\end{solution}

\mycomment

\begin{tabular}{|l|*{\numberofquestions}{c|}c|}\hline
	Вопрос &
	\ForEachQuestion{\QuestionNumber{#1}\iflastquestion{}{&}} &
	Всего \\ \hline
	Баллы &
	\ForEachQuestion{\GetQuestionProperty{points}{#1}\iflastquestion{}{&}} &
	\pointssum* \\ \hline
   Набрано &
   \ForEachQuestion{\iflastquestion{}{&}} & \\ \hline
\end{tabular}
\end{document}
