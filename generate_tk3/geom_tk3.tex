\documentclass[a4paper, 12pt]{article}
\usepackage{cmap}					% поиск в PDF
\usepackage{mathtext} 				% русские буквы в формулах
\usepackage[T2A]{fontenc}			% кодировка
\usepackage[utf8]{inputenc}			% кодировка исходного текста
\usepackage[english,russian]{babel}	% локализация и переносы
\usepackage{indentfirst}
\frenchspacing
\usepackage{geometry}
\geometry{left=3cm, right=1.5cm, top=2cm, bottom=1.5cm}
\usepackage{ifthen}
\usepackage{sagetex}
\usepackage{soul}
\newcounter{zno} % счетчик номеров задач
%%% Дополнительная работа с математикой
\usepackage{amsmath,amsfonts,amssymb,amsthm,mathtools,bm} %
\usepackage{icomma} % "Умная" запятая: $0,2$ --- число, $0, 2$ --- перечисление
\usepackage{etoolbox}
\usepackage{enumitem}
\usepackage{lastpage}
\usepackage{fancyhdr} % Колонтитулы
\pagestyle{fancy}
\renewcommand{\headrulewidth}{0.5pt}  % Толщина линейки, отчеркивающей верхний колонтитул
\renewcommand{\footrulewidth}{0.5pt}
\lfoot{\textit{В.~Н. Кожухова}}
\rfoot{%Иванов Иван Иванович
	\input{\jobname.txt}%
	}
%\rhead{Верхний правый}
\chead{\textbf{ДИТИ НИЯУ МИФИ}, осенний семестр 2017-2018 уч. года}
%\lhead{Верхний левый}
%\cfoot{c. \thepage \; из \pageref{LastPage}} % По умолчанию здесь номер страницы




\makeatletter
\AddEnumerateCounter{\asbuk}{\russian@alph}{щ} % русские буквы в списках
\makeatother

\usepackage{exsheets}
\DeclareTranslation{Russian}{exsheets-exercise-name}{Задача}
\DeclareTranslation{Russian}{exsheets-question-name}{Вопрос}
\DeclareTranslation{Russian}{exsheets-solution-name}{Ответ}
\SetupExSheets[points]{name=б.}

%\def\teachers{}
\newcommand{\mycomment}{\ifdef{\teachers}{\printsolutions}{}}	

\begin{document}
\thispagestyle{empty}
%\maketitle

\vspace{-10ex}
	
\begin{center}
\Huge{Аналитическая геометрия}\\
\vspace{-3ex}
\section*{Текущий контроль 3}
\end{center}
\vspace{-2ex}	
\textit{Стоимость: \pointssum* б.}
\begin{flushleft}
Направление подготовки: 09.03.02 Информационные системы и технологии %03.03.02 Физика %14.03.02 Ядерные физика и технологии 

ФИО студента: %Иванов Иван Иванович
 \input{\jobname.txt}

Преподаватель: \textit{В.",Н. Кожухова}	
\end{flushleft}	

\subsection*{Прямая на плоскости}
\begin{sagesilent}
# генерируем точки М1 и М2
M1 = vector([randint(-7, 7) for i in range(2)])
M2 = vector([randint(-4, 4) for i in range(2)])
# если направляющий вектор нулевой, перегенерируем
while M1==M2:
 M2 = vector([randint(-4, 4) for i in range(2)]) 
# генерируем угол альфа
S3 = FiniteEnumeratedSet([30,45,60,90,135,120,150]) # углы в градусах
ang = S3.random_element() # случайный угол
# направляющий вектор
M1M2=M2-M1
var('x,y,t')
# если в уравнении нет переменной y
if M1M2[0]==0:
   eqline=x-M1[0]==0 # общее
   eqparam=[solve(x-M1[0]==M1M2[0]*t,x),solve(y-M1[1]==M1M2[1]*t,y)] # параметрическое
   eqang=eqline # с угловым коэффициентом
   ccoef=-M1[0] # константа в уравнении
   if M1[0]==0:
     eqotr=eqline # в отрезках чтоб не было деления на ноль
   else:
     eqotr=x/M1[0]==1 # в отрезках
else :
   # если в уравнении нет переменной x
   if M1M2[1]==0:
      eqline=y-M1[1]==0 # общее
      eqparam=[solve(x-M1[0]==M1M2[0]*t,x),solve(y-M1[1]==M1M2[1]*t,y)] # параметрическое
      eqang=solve(eqline,y)[0] # с угловым коэффициентом
      ccoef=-M1[1] # константа в уравнении
      if M1[1]==0:
        eqotr=eqline  # в отрезках чтоб не было деления на ноль
      else:
        eqotr=y/M1[1]==1 # в отрезках
   else :
      # если в уравнении есть обе переменных x и y
      eqcan=(x-M1[0])/M1M2[0]==(y-M1[1])/M1M2[1] # каноническое
      eqparam=[eqcan.left()==t, eqcan.right()==t]
      eqparam=(solve(eqparam,x,y))[0] # параметрическое
      eqcan2=eqcan.subtract_from_both_sides(eqcan.right())
      eqline=eqcan2.multiply_both_sides(lcm(eqcan.left().denominator(), eqcan.right().denominator())) # общее
      eqang=solve(eqline,y)[0] # с угловым коэффициентом
      # коэффициенты уравнения
      ccoef=eqline.left().coefficient(x,0).coefficient(y,0)
      acoef=eqline.left().coefficient(x,1)
      bcoef=eqline.left().coefficient(y,1)
      # проверка константа в уравнении = 0
      if ccoef==0:
         eqotr=eqline # в отрезках чтоб не было деления на ноль
      else:
         eqotr=x/(-ccoef/acoef)+y/(-ccoef/bcoef)==1 # в отрезках
# вектор нормали
nl=vector([eqline.left().coefficient(x,1), eqline.left().coefficient(y,1)])
# делим на +- длину нормального вектора
if ccoef>0:
   eqnorm=eqline.divide_both_sides(-nl.norm())
else:
   eqnorm=eqline.divide_both_sides(nl.norm())
#  направляющие косинусы 
cosline=vector([eqnorm.left().coefficient(x,1), eqnorm.left().coefficient(y,1)])
#  прямая через М1 под углом альфа к Ох
if ang==90:
   eqang2=x==M1[0]
else:
   eqang2=y-M1[1]==tan(ang*pi/180)*(x-M1[0])
   eqang2=solve(eqang2,y)[0]
# треугольник АВС
crossm = zero_matrix(2,2)
while (det(crossm)==0) or (abs(det(crossm)/2) < 8) : 
  A = vector([randint(-7, 7) for i in range(2)])
  B = vector([randint(-4, 4) for i in range(2)])
  C = vector([randint(-10, 10) for i in range(2)])
  AB = B-A
  AC = C-A
  BC = C-B
  crossm = matrix([AB, AC])
chv=randint(1,3) # первая случайная вершина vertex1
ch1=randint(1,2) # вторая случайная вершина vertex2
# присваиваем координаты выбранным вершинам
if chv == 1 :
 vertex = r"A"
 V1=A
 if ch1==1:
  vertex2=r"B"
  V2=B
  vertex3=r"C"
  V3=C
 else: 
  vertex2=r"C"
  V2=C
  vertex3=r"B"
  V3=B
else :
 if chv == 2 :
  vertex = r"B"
  V1=B
  if ch1==1:
   vertex2=r"A"
   V2=A
   vertex3=r"C"
   V3=C
  else: 
   vertex2=r"C"
   V2=C
   vertex3=r"A"
   V3=A
 else :
  vertex = r"C"
  V1=C
  if ch1==1:
   vertex2=r"B"
   V2=B
   vertex3=r"A"
   V3=A
  else: 
   vertex2=r"A"
   V2=A
   vertex3=r"B"
   V3=B
# генерируем L=line2
a1=0
b1=0
c1=0
# чтобы не было нулевых коэф-в и точки АВС не лежали бы на L
while (a1==0) or (b1==0) or (c1==0) or (a1*A[0]+b1*A[1]+c1==0) or (a1*B[0]+b1*B[1]+c1==0) or (a1*C[0]+b1*C[1]+c1==0) :
 a1=randint(-10, 10)
 b1=randint(-10, 10)
 c1=randint(-10, 10)
 line2=a1*x+b1*y+c1==0
# уравнения через вершину параллельно и перпендикулярно line2=L
nLx=line2.left().coefficient(x,1)
nLy=line2.left().coefficient(y,1)
# параллельно
eqLparal=nLx*(x-V1[0])+nLy*(y-V1[1])==0
# перпендикулярно
eqLorthog=(x-V1[0])/nLx-(y-V1[1])/nLy==0
eqLorthog=eqLorthog.multiply_both_sides(eqLorthog.left().denominator())
# точка пересечения высот
# высота через вершину V1
h1=(V3[0]-V2[0])*(x-V1[0])+(V3[1]-V2[1])*(y-V1[1])==0
# высота через вершину V2
h2=(V3[0]-V1[0])*(x-V2[0])+(V3[1]-V1[1])*(y-V2[1])==0
# точка пересечения высот
h1h2=solve([h1,h2],x,y)
hh=vector([h1h2[0][0].right(), h1h2[0][1].right()])
# длина перпендикуляра из V1 на медиану из V2
# направляющий вектор медианы
smedx=(V3[0]+V1[0])/2-V2[0]
smedy=(V3[1]+V1[1])/2-V2[1]
# если одна из координат напр. вектора медианы = 0
if smedx==0:
 vert2med=x-V2[0]==0
 vert1h=(y-V1[1])==0
else:
 if smedy==0:
  vert2med=y-V2[1]==0
  vert1h=(x-V1[0])==0
 else:
  vert2med=(x-V2[0])/(smedx)-(y-V2[1])/(smedy)==0
  vert1h=(x-V1[0])/vert2med.left().coefficient(x,1)-(y-V1[1])/vert2med.left().coefficient(y,1)==0
vert2med=vert2med.multiply_both_sides(vert2med.left().denominator())
vert1h=vert1h.multiply_both_sides(vert1h.left().denominator())
# точка пересечения медианы и перпендикуляра
medh=solve([vert2med,vert1h],x,y)
medh=vector([medh[0][0].right(), medh[0][1].right()])
# длина перпендикуляра
perp=(medh-V1).norm()
# уравнения биссектрис угла между прямыми V1V2 и V1V3
# V1V2
V1V2=(x-V1[0])*(V2[1]-V1[1])-(y-V1[1])*(V2[0]-V1[0])==0
# V1V3
V1V3=(x-V1[0])*(V3[1]-V1[1])-(y-V1[1])*(V3[0]-V1[0])==0
# вектор нормали V1V2
nV1V2=vector([V1V2.left().coefficient(x,1), V1V2.left().coefficient(y,1)])
# вектор нормали V1V3
nV1V3=vector([V1V3.left().coefficient(x,1), V1V3.left().coefficient(y,1)])
# делим на +- длину нормального вектора
if V1V2.left().coefficient(x,0).coefficient(y,0)>0:
 v1v2norm=V1V2.divide_both_sides(-nV1V2.norm())
else:
 v1v2norm=V1V2.divide_both_sides(nV1V2.norm())
if V1V3.left().coefficient(x,0).coefficient(y,0)>0:
 v1v3norm=V1V3.divide_both_sides(-nV1V3.norm())
else:
 v1v3norm=V1V3.divide_both_sides(nV1V3.norm())
denom=gcd((nV1V2.norm(),nV1V3.norm()))
biss1=v1v2norm.left()-v1v3norm.left()
biss2=v1v2norm.left()+v1v3norm.left()
biss1=biss1.canonicalize_radical().mul(denom)
biss2=biss2.canonicalize_radical().mul(denom)
biss1=biss1.expand().collect(x).collect(y)
biss2=biss2.expand().collect(x).collect(y)
biss1=biss1.mul(biss1.denominator())
biss2=biss2.mul(biss2.denominator())
if (sign(biss1.subs(x==V2[0],y==V2[1])))==(sign(biss1.subs(x==V3[0],y==V3[1]))):
 bissstr = r"внешнего"
else:
 bissstr = r"внутреннего"
# найти точку симметричную V1 относительно V2V3
# V2V3
V2V3=(x-V2[0])*(V3[1]-V2[1])-(y-V2[1])*(V3[0]-V2[0])==0
aV2V3=V2V3.left().coefficient(x,1)
bV2V3=V2V3.left().coefficient(y,1)
cV2V3=V2V3.left().coefficient(x,0).coefficient(y,0)
eq1=bV2V3*(x-V1[0])-aV2V3*(y-V1[1])==0
eq2=(aV2V3^2+bV2V3^2)*(x-V1[0])+2*aV2V3*(aV2V3*V1[0]+bV2V3*V1[1]+cV2V3)==0
eq3=(aV2V3^2+bV2V3^2)*(y-V1[1])+2*bV2V3*(aV2V3*V1[0]+bV2V3*V1[1]+cV2V3)==0
# точкa симметричная V1 относительно V2V3
pp=vector([solve([eq1,eq2,eq3],x,y)[0][0].right(),solve([eq1,eq2,eq3],x,y)[0][1].right()])
# график треугольника
P1 = {'A':A,'B':B,'C':C}
P2 = {'A':A,'B':B,'C':C, 'H':hh}
P3 = {'A':A,'B':B,'C':C,'Q':pp}
# настройка графика - минимум-максимум по отображаемым точкам
def Min_Max_Values(P) :
    x=vector(QQ,[i for i in range(len(P.values()))])
    y=vector(QQ,[i for i in range(len(P.values()))])
    for i in range(len(P.values())):
        x[i]=P.values()[i][0]
        y[i]=P.values()[i][1]
    xmin=min(x)
    xmax=max(x)
    if sign(xmin) == sign(xmax):
       if sign(xmin)==-1:
          xmax=abs(xmax)
       else :
          if sign(xmin)==0:
             xmin=-10
             xmax=10
          else:
             xmin=-xmin
    ymin=min(y)
    ymax=max(y)
    if sign(ymin) == sign(ymax):
       if sign(ymin)==-1:
          ymax=abs(ymax)
       else :
          if sign(ymin)==0:
             ymin=-10
             ymax=10
          else:
             ymin=-ymin  
    res = vector(QQ,[xmin,xmax,ymin,ymax])
    return res
# min max graph 1-3
mmg=Min_Max_Values(P1)
xmin=mmg[0]
xmax=mmg[1]
ymin=mmg[2]
ymax=mmg[3]
# параллельно и перпендикулярно L
G1=points(P1.values(), size=50, color='black', figsize=4, aspect_ratio=1, xmin=1.5*xmin, xmax=1.5*xmax, ymin=1.5*ymin, ymax=1.5*ymax)+line((A,B), color='black')+line((A,C), color='black')+line((B,C), color='black')+plot(solve(eqLparal,y)[0].right(), (x,1.5*xmin,1.5*xmax))+plot(solve(eqLorthog,y)[0].right(), (x,1.5*xmin,1.5*xmax)) + plot(solve(line2,y)[0].right(), (x,1.5*xmin,1.5*xmax), color='red', legend_label='L')
i=0
for p in P1.keys():
  G1 += text('  %s'%p,P1.values()[i],horizontal_alignment='left',color='black')
  i=i+1
# высоты и их точка пересечения
mmg=Min_Max_Values(P2)
xmin=mmg[0]
xmax=mmg[1]
ymin=mmg[2]
ymax=mmg[3]
G2=points(P2.values(), size=50, color='black', figsize=4, aspect_ratio=1, xmin=1.5*xmin, xmax=1.5*xmax, ymin=1.5*ymin, ymax=1.5*ymax)+line((A,B), color='black')+line((A,C), color='black')+line((B,C), color='black')+plot(solve(h1,y)[0].right(), (x,1.5*xmin,1.5*xmax), legend_label='h1')+plot(solve(h2,y)[0].right(), (x,1.5*xmin,1.5*xmax), legend_label='h2')
i=0
for p in P2.keys():
  G2 += text('  %s'%p,P2.values()[i],horizontal_alignment='left',color='black')
  i=i+1
# медиана и перпендикуляр
#G3
# биссектрисы и симметричная точка
mmg=Min_Max_Values(P3)
xmin=mmg[0]
xmax=mmg[1]
ymin=mmg[2]
ymax=mmg[3]
G4=points(P3.values(), size=50, color='black', figsize=4, aspect_ratio=1, xmin=1.5*xmin, xmax=1.5*xmax, ymin=1.5*ymin, ymax=1.5*ymax)+line((A,B), color='black')+line((A,C), color='black')+line((B,C), color='black')+plot(solve(biss1==0,y)[0].right(), (x,1.5*xmin,1.5*xmax), legend_label='b1')+plot(solve(biss2==0,y)[0].right(), (x,1.5*xmin,1.5*xmax), color='red', legend_label='b2')
i=0
for p in P3.keys():
 G4 += text('  %s'%p,P3.values()[i],horizontal_alignment='left',color='black')
 i=i+1
 # генерируем L1 и L2
a1=0
b1=0
c1=0
while (a1==0) or (b1==0) or (c1==0) :
  a1=randint(-10, 10)
  b1=randint(-10, 10)
  c1=randint(-10, 10)
lineL1=a1*x+b1*y+c1==0
a2=0
b2=0
c2=0
# они не должны быть параллельны
while (a2==0) or (b2==0) or (c2==0) or (a1*b2-a2*b1==0):
  a2=randint(-10, 10)
  b2=randint(-10, 10)
  c2=randint(-10, 10)
lineL2=a2*x+b2*y+c2==0
# генерируем N
N = vector([0,0])
# он не должен на них лежать
while (N==vector([0,0])) or (lineL1.left().subs(x==N[0],y==N[1])==0) or (lineL2.left().subs(x==N[0],y==N[1])==0):
   N = vector([randint(-10, 10) for i in range(2)])
# расстояние d
d=randint(1, 10)
# K - точка пересечения L1 и L2
K=vector([solve([lineL1,lineL2],x,y)[0][0].right(),solve([lineL1,lineL2],x,y)[0][1].right()])
# N - середина KM, находим M
M=vector([2*N[0]-K[0],2*N[1]-K[1]])
# сторона парал-ма L3 - через M // L1
lineL3=a1*(x-M[0])+b1*(y-M[1])==0
# сторона парал-ма L4 - через M // L3
lineL4=a2*(x-M[0])+b2*(y-M[1])==0
# диагональ d1 =KM
lined1=(x-K[0])*(M[1]-K[1])-(y-K[1])*(M[0]-K[0])==0
lined1=lined1.multiply_both_sides(lined1.left().denominator())
lined1=lined1.divide_both_sides(gcd((lined1.left().coefficient(x,1), lined1.left().coefficient(y,1), lined1.left().coefficient(x,0).coefficient(y,0))))
# R - точка пересечения L1 и L4
R=vector([solve([lineL1,lineL4],x,y)[0][0].right(),solve([lineL1,lineL4],x,y)[0][1].right()])
# диагональ d2 = NR
lined2=(x-N[0])*(R[1]-N[1])-(y-N[1])*(R[0]-N[0])==0
lined2=lined2.multiply_both_sides(lined2.left().denominator())
lined2=lined2.divide_both_sides(gcd((lined2.left().coefficient(x,1), lined2.left().coefficient(y,1), lined2.left().coefficient(x,0).coefficient(y,0))))
# через N перпендикулярно L1 = сторона NE
lineNE=(x-N[0])*b1-(y-N[1])*a1==0
lineNE=lineNE.multiply_both_sides(lineNE.left().denominator())
lineNE=lineNE.divide_both_sides(gcd((lineNE.left().coefficient(x,1), lineNE.left().coefficient(y,1), lineNE.left().coefficient(x,0).coefficient(y,0))))
# через N перпендикулярно L2 = сторона NF
lineNF=(x-N[0])*b2-(y-N[1])*a2==0
lineNF=lineNF.multiply_both_sides(lineNF.left().denominator())
lineNF=lineNF.divide_both_sides(gcd((lineNF.left().coefficient(x,1), lineNF.left().coefficient(y,1), lineNF.left().coefficient(x,0).coefficient(y,0))))
# E - точка пересечения L2 и NE
E=vector([solve([lineL2,lineNE],x,y)[0][0].right(),solve([lineL2,lineNE],x,y)[0][1].right()])
# F - точка пересечения L1 и NF
F=vector([solve([lineL1,lineNF],x,y)[0][0].right(),solve([lineL1,lineNF],x,y)[0][1].right()])
# сторона EF
lineEF=(x-E[0])*(F[1]-E[1])-(y-E[1])*(F[0]-E[0])==0
lineEF=lineEF.multiply_both_sides(lineEF.left().denominator())
lineEF=lineEF.divide_both_sides(gcd((lineEF.left().coefficient(x,1), lineEF.left().coefficient(y,1), lineEF.left().coefficient(x,0).coefficient(y,0))))
# для графика
l1G=solve(lineL1,y)[0].right()
l2G=solve(lineL2,y)[0].right()
lNEG=solve(lineNE,y)[0].right()
lNFG=solve(lineNF,y)[0].right()
lEFG=solve(lineEF,y)[0].right()
# график этих прямых и точек N, E, F
P4 = {'N':N,'E':E,'F':F}
mmg=Min_Max_Values(P4)
xmin=mmg[0]
xmax=mmg[1]
ymin=mmg[2]
ymax=mmg[3]
G5=plot(l1G, (x,1.5*xmin,1.5*xmax), thickness=0.5, color='red', ymin=1.5*ymin, ymax=1.5*ymax, figsize=4, aspect_ratio=1, legend_label='L1') + plot(l2G, (x,1.5*xmin,1.5*xmax), thickness=0.5, color='green', ymin=1.5*ymin, ymax=1.5*ymax, legend_label='L2') + plot(lNEG, (x,1.5*xmin,1.5*xmax), thickness=0.5, color='black', ymin=1.5*ymin, ymax=1.5*ymax) + plot(lNFG, (x,1.5*xmin,1.5*xmax), thickness=0.5, color='black',ymin=1.5*ymin, ymax=1.5*ymax)+plot(lEFG, (x,1.5*xmin,1.5*xmax), thickness=0.5, color='black', ymin=1.5*ymin, ymax=1.5*ymax)+ points(P4.values(), color='black', ymin=1.5*ymin, ymax=1.5*ymax, xmin=1.5*xmin, xmax=1.5*xmax)
i=0
for p in P4.keys():
  G5 += text('  %s'%p,P4.values()[i],horizontal_alignment='left',color='black')
  i=i+1
# расстояние от точки N до прямой L1
# вектор нормали
nl1=vector([a1, b1])
# делим на +- длину нормального вектора
if c1>0:
  lineL1norm=lineL1.divide_both_sides(-nl1.norm())
else:
  lineL1norm=lineL1.divide_both_sides(nl1.norm())
# расстояние
dNL1=abs(lineL1norm.left().subs(x==N[0], y==N[1]))
# вектор нормали
nl2=vector([a2, b2])
# делим на +- длину нормального вектора
if c2>0:
  lineL2norm=lineL2.divide_both_sides(-nl2.norm())
else:
  lineL2norm=lineL2.divide_both_sides(nl2.norm())
if sign(lineL2norm.left().subs(x==N[0], y==N[1])) == sign(lineL2norm.left().subs(x==0, y==0)):
  intercept=r"не"
else:
  intercept=r""
# уравнения прямых паралелльных L1
var('cc')
newL=a1*x+b1*y+cc==0
newLNorm=newL.divide_both_sides(sqrt(a1^2+b1^2))
U=vector([0,solve(lineL1.subs(x==0),y)[0].right()])
eq1=newLNorm.left().subs(x==U[0],y==U[1])==d
eq2=newLNorm.left().subs(x==U[0],y==U[1])==-d
cc1=solve(eq1,cc)[0].right()
cc2=solve(eq2,cc)[0].right()
G6=plot(solve(lineL1,y)[0].right(), (x,-10,10), aspect_ratio=1, color='red', figsize=4)+plot(solve(newL(cc=cc1),y)[0].right(), (x,-10,10), aspect_ratio=1, figsize=4)+plot(solve(newL(cc=cc2),y)[0].right(), (x,-10,10), aspect_ratio=1, figsize=4)
# квадрат
# сторона Lp через N перпендикулярно L1 ( = NE)
lineLp=lineNE
ap=lineLp.left().coefficient(x,1)
bp=lineLp.left().coefficient(y,1)
# сторона Lq через N параллельно L1 
lineLq=a1*(x-N[0])+b1*(y-N[1])==0
# стороны Lr и Lt параллельно Lp на расстоянии dNL1
lineLrt=ap*x+bp*y+cc==0
lineLrtNorm=lineLrt.divide_both_sides(sqrt(ap^2+bp^2))
eq1=lineLrtNorm.left().subs(x==N[0],y==N[1])==dNL1
eq2=lineLrtNorm.left().subs(x==N[0],y==N[1])==-dNL1
cc3=solve(eq1,cc)[0].right()
cc4=solve(eq2,cc)[0].right()
G7=plot(solve(lineL1,y)[0].right(), (x,-15,15), aspect_ratio=1, color='red', figsize=4, legend_label='L1', ymin=-20, ymax=20)+plot(solve(lineLrt(cc=cc3),y)[0].right(), (x,-10,10), aspect_ratio=1, figsize=4)+plot(solve(lineLrt(cc=cc4),y)[0].right(), (x,-10,10), aspect_ratio=1, figsize=4)+plot(solve(lineLp,y)[0].right(), (x,-10,10), aspect_ratio=1, color='green', figsize=4)+plot(solve(lineLq,y)[0].right(), (x,-10,10), aspect_ratio=1, color='purple', figsize=4)+ point(N, color='black', xmin=-10, xmax=10)+text('  N',N,horizontal_alignment='left',color='black')
# угол между прямыми
angL1L2=arccos(abs(nl1.dot_product(nl2)/(nl1.norm()*nl2.norm())))
# отклонения точек O и N от L1 и L2
deltaOL1=lineL1norm.left().subs(x==0,y==0)
deltaOL2=lineL2norm.left().subs(x==0,y==0)
deltaNL1=lineL1norm.left().subs(x==N[0],y==N[1])
deltaNL2=lineL2norm.left().subs(x==N[0],y==N[1])
# angO = 1 если O в остром углу и = 2 если в тупом
n1orient=vector([lineL1norm.left().coefficient(x,1), lineL1norm.left().coefficient(y,1)])
n2orient=vector([lineL2norm.left().coefficient(x,1), lineL2norm.left().coefficient(y,1)])
if n1orient.dot_product(n2orient)>0:
   angO = -1 # в тупом
else:
   angO = 1 # в остром, в т.ч. в прямом
# взаимное расположение O и N
if (sign(deltaOL1)==-sign(deltaNL1)) and (sign(deltaOL2)==-sign(deltaNL2)):
   angON = 1 # вертикальные углы
   angN = angO # вертикальные углы
else:
   if (sign(deltaOL1)==sign(deltaNL1)) and (sign(deltaOL2)==sign(deltaNL2)):
      angON = 2 # один и тот же угол
      angN = angO
   else:
      angON = 3 # смежные углы
      angN = -angO
if angN==1:
   angNstr=r"остром"
else:
   angNstr=r"тупом"
if angON==1:
   angONstr=r"вертикальном"
else:
   if angON==2:
      angONstr=r"одном"
   else:
      angONstr=r"смежном"
# уравнения биссектрис угла между прямыми L1 и L2
denom=gcd((nl1.norm(),nl2.norm()))
biss3=lineL1norm.left()-lineL2norm.left()
biss4=lineL1norm.left()+lineL2norm.left()
biss3=biss3.canonicalize_radical().mul(denom)
biss4=biss4.canonicalize_radical().mul(denom)
biss3=biss3.expand().collect(x).collect(y)
biss4=biss4.expand().collect(x).collect(y)
biss3=biss3.mul(biss3.denominator())
biss4=biss4.mul(biss4.denominator())
G8=plot(l1G, (x,-10,10), color='red', figsize=4, aspect_ratio=1, legend_label='L1', ymin=-15, ymax=15) + plot(l2G, (x,-10,10), color='green', legend_label='L2') + plot(solve(biss3==0,y)[0].right(), (x,-10,10), legend_label='b1') + plot(solve(biss4==0,y)[0].right(), (x,-10,10), color='black', legend_label='b2')+ point(N, color='black', xmin=-10, xmax=10)+text('  N',N,horizontal_alignment='left',color='black')
\end{sagesilent}

\vspace{2ex}	
\begin{question}{1}
Две точки на плоскости заданы координатами: $M_1\sage{M1}$ и $M_2\sage{M2}$,\\ $\angle\alpha=\sage{ang}^\circ$ -- некоторый угол.

Составить:
\begin{enumerate}
	\item Уравнение прямой, проходящей через точку $M_1$ и образующей с осью абсцисс $\angle\alpha$.
	\item Уравнение прямой на плоскости, проходящей через
	точки $M_1$ и $M_2$. Записать это уравнение в следующих видах:
	\begin{enumerate}[label=\asbuk*),ref=\asbuk*]
		\item общем;
		\item каноническом;
		\item параметрическом;
		\item с угловым коэффициентом;
		\item в отрезках;
		\item нормальном. 
	\end{enumerate}
	Указать направляющие косинусы данной прямой.
\end{enumerate}
\end{question}
\begin{solution}
через точку под углом: \ensuremath{\displaystyle{\sage{eqang2}}} 

общее уравнение: \ensuremath{\displaystyle{\sage{eqline}}}, 

направляющий вектор: \ensuremath{\displaystyle{M_1M_2= \sage{M1M2}}}, 

каноническое: \ensuremath{\displaystyle{\dfrac{\sage{x-M1[0]}}{\sage{M1M2[0]}}=\dfrac{\sage{y-M1[1]}}{\sage{M1M2[1]}}}}, 

параметрическое: \ensuremath{\displaystyle{\sage{eqparam}}}

с угловым коэффициентом: \ensuremath{\displaystyle{\sage{eqang}}},

в отрезках: \ensuremath{\displaystyle{\sage{eqotr}}}, 

нормальное: \ensuremath{\displaystyle{\sage{eqnorm}}}, 

направляющие косинусы: \ensuremath{\displaystyle{\sage{cosline}}}
\end{solution}
\begin{question}{1}\label{ztriag}
Даны вершины треугольника $ABC$ и прямая $L$: \[A\sage{A}, B\sage{B}, C\sage{C}, L: \sage{line2}\]
\begin{enumerate}
	\item Составить уравнения прямой $L_1$, проходящей через точку $\sagestr{vertex}$ параллельно прямой $L$ и прямой $L_2$, проходящей через точку $\sagestr{vertex}$ перпендикулярно прямой $L$.  
	\item Определить точку пересечения высот треугольника.
	\item Вычислить длину перпендикуляра, опущенного из вершины $\sagestr{vertex}$ на медиану, проведенную из вершины $\sagestr{vertex2}$.
\end{enumerate}
\end{question}
\begin{solution}
\sageplot{G1} \sageplot{G2}\ 
 
параллельно: \ensuremath{\displaystyle{\sage{eqLparal}}}

перпендикулярно: \ensuremath{\displaystyle{\sage{eqLorthog}}}

точка пересечения высот: \ensuremath{\displaystyle{\sage{h1}, \sage{h2}, \sage{h1h2}}}

медиана: \ensuremath{\displaystyle{\sage{vert2med}}}, перпендикуляр: \ensuremath{\displaystyle{\sage{vert1h}}},

длина перпендикуляра: \ensuremath{\displaystyle{\sage{perp}}}

\end{solution}

\begin{question}{1}
	По данным задачи \ref{ztriag}:
	\begin{enumerate}
		\item Составить уравнения биссектрис внутреннего и внешнего углов треугольника $ABC$ при вершине $\sagestr{vertex}$. 
		\item Найти координаты точки $Q$, симметричной точке $\sagestr{vertex}$  относительно прямой $\sagestr{vertex2}\sagestr{vertex3}$. 
	\end{enumerate}
\end{question}
\begin{solution}
биссектрисы: 

\ensuremath{\displaystyle{\sage{biss1}=0}} 	

-- биссектриса $\sagestr{bissstr}$ угла

\ensuremath{\displaystyle{\sage{biss2}=0}}

симметричная точка: \ensuremath{\displaystyle{\sage{pp}}}

\sageplot{G4}
\end{solution}
\begin{question}{1}\label{zlines}
Данa точка $N$ и две прямые $L_1$ и $L_2$:
\[N\sage{N}, L_1: \sage{lineL1}, L_2: \sage{lineL2}\]	
\begin{enumerate}
	\item Пусть прямые $L_1$ и $L_2$ -- стороны параллелограмма, а $N$ -- точка пересечения его диагоналей. Написать уравнения остальных сторон и диагоналей параллелограмма.
	\item Составить уравнения сторон треугольника, если одна из его вершин -- точка $N$, а прямые $L_1$ и $L_2$ -- высоты этого треугольника.
	\item Найти расстояние от точки $N$ до прямой $L_1$. 
	\item Написать нормальное уравнение прямой $L_1$. Пересекает прямая $L_2$ отрезок $ON$ или нет? (точка $O$ -- начало координат). 
\end{enumerate}
\end{question}
\begin{solution}
стороны параллелограмма: \ensuremath{\displaystyle{\sage{lineL3}, \sage{lineL4}}}

диагонали параллелограмма: \ensuremath{\displaystyle{\sage{lined1}, \sage{lined2}}}

уравнения сторон треугольника:

 \ensuremath{\displaystyle{\sage{lineNE}, \sage{lineNF}, \sage{lineEF} }}
 
расстояние от $N$ до $L_1$: \ensuremath{\displaystyle{\sage{dNL1}\approx\sage{n(dNL1, digits=4)}}}

нормальное уравнение $L_1$: \ensuremath{\displaystyle{\sage{lineL1norm}}}

$L_2$ $\sagestr{intercept}$ пересекает $ON$

\sageplot{G5} 
\end{solution}	

\begin{question}{1}
	По данным задачи \ref{zlines}:
\begin{enumerate}
	\item Написать уравнения прямых, параллельных $L_1$ и отстоящих от нее на расстоянии $d = \sage{d}$. 
	\item Пусть $L_1$ -- одна из сторон квадрата, а точка $N$ -- его вершина. Написать уравнения остальных сторон квадрата. 
	\item Определить аналитически, какой из углов, тупой или острый, образованный прямыми $L_1$ и $L_2$, содержит точку $N$. Определить величину угла между прямыми $L_1$ и $L_2$. 
	\item Составить уравнение биссектрисы угла между прямыми $L_1$ и $L_2$, смежного с углом, содержащим точку $N$.
\end{enumerate}
\end{question}
\begin{solution}
прямые, параллельные $L_1$ на расстоянии $\sage{d}$: 

\ensuremath{\displaystyle{\sage{newL(cc=cc1)}, \sage{newL(cc=cc2)}}}

стороны квадрата: \ensuremath{\displaystyle{\sage{lineLp}, \sage{lineLq}}}

\ensuremath{\displaystyle{\sage{lineLrt(cc=cc3)}, \sage{lineLrt(cc=cc4)}}}

угол между $L_1$ и $L_2$: \ensuremath{\displaystyle{\sage{angL1L2}\approx \sage{round(angL1L2*180/pi)}^\circ}}	

$N$ расположена в $\sagestr{angNstr}$ углу

$O$ и $N$ находятся в $\sagestr{angONstr}$ углу

биссектрисы углов между $L_1$ и $L_2$: 
 
\ensuremath{\displaystyle{\sage{biss3}=0}} 	

\ensuremath{\displaystyle{\sage{biss4}=0}}

\sageplot{G6} \sageplot{G7} 

\sageplot{G8}
\end{solution}

\mycomment

\begin{tabular}{|l|*{\numberofquestions}{c|}c|}\hline
	Вопрос &
	\ForEachQuestion{\QuestionNumber{#1}\iflastquestion{}{&}} &
	Всего \\ \hline
	Баллы &
	\ForEachQuestion{\GetQuestionProperty{points}{#1}\iflastquestion{}{&}} &
	\pointssum* \\ \hline
	Набрано &
	\ForEachQuestion{\iflastquestion{}{&}} & \\ \hline
\end{tabular}
	
\end{document}
