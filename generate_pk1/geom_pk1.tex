% !TeX program = pdflatex
\documentclass[a4paper, 12pt]{article}
\usepackage{cmap}					% поиск в PDF
\usepackage{mathtext} 				% русские буквы в формулах
\usepackage[T2A]{fontenc}			% кодировка
\usepackage[utf8]{inputenc}			% кодировка исходного текста
\usepackage[english,russian]{babel}	% локализация и переносы
\usepackage{indentfirst}
\frenchspacing
\usepackage{geometry}
\geometry{left=3cm, right=1.5cm, top=2cm, bottom=1.5cm}
\usepackage{ifthen}
\usepackage{sagetex}
\usepackage{soul}

%%% Дополнительная работа с математикой
\usepackage{amsmath,amsfonts,amssymb,amsthm,mathtools,bm} %
\usepackage{icomma} % "Умная" запятая: $0,2$ --- число, $0, 2$ --- перечисление
\usepackage{etoolbox}
\usepackage{enumitem}
\usepackage{verbatim}


\usepackage{lastpage}
\usepackage{fancyhdr} % Колонтитулы
\pagestyle{fancy}
\renewcommand{\headrulewidth}{0.5pt}  % Толщина линейки, отчеркивающей верхний колонтитул
\renewcommand{\footrulewidth}{0.5pt}
\lfoot{\textit{В.~Н. Кожухова}}
\rfoot{Вариант № 
%	\input{\jobname.txt}
	}
%\rhead{Верхний правый}
\chead{\textbf{ДИТИ НИЯУ МИФИ}, осенний семестр 2017-2018 уч. года}
%\lhead{Верхний левый}
%\cfoot{c. \thepage \; из \pageref{LastPage}} % По умолчанию здесь номер страницы


\makeatletter
\AddEnumerateCounter{\asbuk}{\russian@alph}{щ}
\makeatother
%\usepackage[explicit]{titlesec}

\usepackage{exsheets}
\DeclareTranslation{Russian}{exsheets-exercise-name}{Задача}
\DeclareTranslation{Russian}{exsheets-question-name}{Вопрос}
\DeclareTranslation{Russian}{exsheets-solution-name}{Ответ}


\def\teachers{}
\newcommand{\mycomment}{\ifdef{\teachers}{\printsolutions}{}}

\begin{document}
\thispagestyle{empty}
\vspace{-10ex}

\begin{center}
	\Huge{Аналитическая геометрия}\\
	\vspace{-3ex}
	\section*{Промежуточный контроль 1}
	\vspace{-2ex}
	\subsection*{Вариант №} 
		%\input{\jobname.txt}
		\vspace{-1ex}
	\normalsize\today		
		
\end{center}
\vspace{-2ex}	
\textit{Стоимость: 15 баллов}
\begin{flushleft}
	Направление подготовки: 03.03.02 Физика
	
	Преподаватель: \textit{В.",Н. Кожухова}
	
	ФИО студента: \rule{\linewidth}{0.5 pt}	
\end{flushleft}	

\begin{sagesilent}
# Разложение вектора по векторам
p11=zero_vector(3)
q11=p11
r11=p11
mpqr11 = matrix([p11, q11, r11])
while p11==zero_vector(3) or q11==zero_vector(3) or r11==zero_vector(3) or det(mpqr11)==0:
    p11 = vector([randint(-7, 7) for i in range(3)])
    q11 = vector([randint(-3, 3) for i in range(3)])
    r11 = vector([randint(-10, 10) for i in range(3)])
    mpqr11 = matrix([p11, q11, r11])
coord11 = vector([0,0,0])
while coord11 == vector([0,0,0]) : 
    coord11 = vector([randint(-4, 4) for i in range(3)])
x11 = coord11[0]*p11 + coord11[1]*q11 + coord11[2]*r11
# Найти периметр и площадь треугольника на векторах a и b
var('u,v,m,n')
u12=zero_vector(3) 
v12=u12
while u12==zero_vector(3) or v12==zero_vector(3) or u12.cross_product(v12)==zero_vector(3):
	 u12 = vector([randint(-3, 4) for i in range(3)])
	 v12 = vector([randint(-4, 3) for i in range(3)])
basisuv12 = vector([u,v])
acoord12=zero_vector(2)
bcoord12=zero_vector(2)
mab12 = matrix([acoord12, bcoord12])
while acoord12[0]==0 or acoord12[1]==0 or bcoord12[0]==0 or bcoord12[1]==0 or det(mab12)==0:
	 acoord12 = vector([randint(-3, 4) for i in range(2)])
	 bcoord12 = vector([randint(-4, 3) for i in range(2)])
	 mab12 = matrix([acoord12, bcoord12])
a12print=basisuv12.dot_product(acoord12)
b12print=basisuv12.dot_product(bcoord12)
a12=acoord12[0]*u12+acoord12[1]*v12
b12=bcoord12[0]*u12+bcoord12[1]*v12
S12=(a12.cross_product(b12)).norm()/2
P12=a12.norm()+b12.norm()+(b12-a12).norm()
# угол между диагоналями треугольника
normm13=randint(1,4)
normn13=randint(1,3)
Set13 = FiniteEnumeratedSet([30,45,60,90,135,120,150]) # углы в градусах
ang13 = Set13.random_element() # случайный угол
basismn13 = vector([m,n])
acoord13=zero_vector(2)
bcoord13=zero_vector(2)
mab13 = matrix([acoord13, bcoord13])
while acoord13==zero_vector(2) or bcoord13==zero_vector(2) or det(mab13)==0:
    acoord13 = vector([randint(-3, 4) for i in range(2)])
    bcoord13 = vector([randint(-4, 3) for i in range(2)])
    mab13 = matrix([acoord13, bcoord13])
a13print=basismn13.dot_product(acoord13)
b13print=basismn13.dot_product(bcoord13)
d1coord13=acoord13+bcoord13
d2coord13=acoord13-bcoord13
d1d2dot13=d1coord13[0]*d2coord13[0]*normm13^2+(d1coord13[0]*d2coord13[1]+d1coord13[1]*d2coord13[0])*normm13*normn13*cos(ang13*pi/180)+d1coord13[1]*d2coord13[1]*normn13^2
d1norm13=sqrt(d1coord13[0]^2*normm13^2+d1coord13[1]^2*normn13^2+2*d1coord13[0]*d1coord13[1]*normm13*normn13*cos(ang13*pi/180))
d2norm13=sqrt(d2coord13[0]^2*normm13^2+d2coord13[1]^2*normn13^2+2*d2coord13[0]*d2coord13[1]*normm13*normn13*cos(ang13*pi/180))
d1d2ang13=d1d2dot13/(d1norm13*d2norm13)
# длина вектора и направляющие косинусы
A14=zero_vector(3)
B14=A14
C14=A14
mABC14 = matrix([A14, B14, C14])
while A14==zero_vector(3) or B14==zero_vector(3) or C14==zero_vector(3) or det(mABC14)==0:
    A14 = vector([randint(-7, 7) for i in range(3)])
    B14 = vector([randint(-3, 3) for i in range(3)])
    C14 = vector([randint(-5, 5) for i in range(3)])
    mABC14 = matrix([A14, B14, C14])
AC14=C14-A14
AB14=B14-A14
BC14=C14-B14
Set14 = {'AB':AB14,'BC':BC14,'AC':AC14, 'BA':-AB14,'CB':-BC14,'CA':-AC14}
import random
vec114=random.choice(Set14.keys())
vec214=vec114
while vec214==vec114 or Set14[vec114].cross_product(Set14[vec214])==zero_vector(3):
    vec214=random.choice(Set14.keys())
c14=randint(2,3)
p14=c14*(Set14[vec114].cross_product(Set14[vec214]))
p14norm=p14.norm()
p14cos=p14/p14.norm()
# Проверка компланарности векторов
ch15 = randint(1,2) #1 - ЛЗ, #2 ЛНЗ
if ch15==1 :
    a15 = vector([randint(-7, 7) for i in range(3)])
    b15 = vector([randint(-3, 3) for i in range(3)])
    c15 = randint(-5,5)*a15 + randint(-5,5)*b15
else :
    a15 = vector([randint(-7, 7) for i in range(3)])
    b15 = vector([randint(-3, 3) for i in range(3)])
    c15 = vector([randint(-5, 5) for i in range(3)])
abc15 = c15.dot_product(a15.cross_product(b15))  
if abc15 == 0 :
    compl15 = r"компланарны"
else :
    compl15 = r"не компланарны"
\end{sagesilent}
	
\vspace{2ex}

\pgfmathsetseed{723715009%\number\pdfrandomseed
%\myvar
	}
\includequestions[random=5]{geom_pk1_vectors.tex}
%\includequestions[random=1]{geom_pk1_planepolar.tex}
%\includequestions[random=2]{geom_pk1_lines.tex}

\mycomment
\end{document}
