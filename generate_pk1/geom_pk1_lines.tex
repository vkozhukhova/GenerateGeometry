\begin{question}
Даны вершины треугольника $ABC$ и прямая $L$: \[A\sage{A}, B\sage{B}, C\sage{C}, L: \sage{line2}\].
Составить уравнения прямой $L_1$, проходящей через точку $\sagestr{vertex}$ параллельно прямой $L$ и прямой $L_2$, проходящей через точку $\sagestr{vertex}$ перпендикулярно прямой $L$. 
\end{question}
\begin{solution}
параллельно: \ensuremath{\displaystyle{\sage{eqLparal}}}

перпендикулярно: \ensuremath{\displaystyle{\sage{eqLorthog}}}
\end{solution}

\begin{question}
Даны вершины треугольника $ABC$: \[A\sage{A}, B\sage{B}, C\sage{C}\].
Определить точку пересечения его высот.
\end{question}
\begin{solution}
точка пересечения высот: \ensuremath{\displaystyle{\sage{h1}, \sage{h2}, \sage{h1h2}}}
\end{solution}

\begin{question}
Даны вершины треугольника $ABC$: \[A\sage{A}, B\sage{B}, C\sage{C}\].
Вычислить длину перпендикуляра, опущенного из вершины $\sagestr{vertex}$ на медиану, проведенную из вершины $\sagestr{vertex2}$.
\end{question}
\begin{solution}
медиана: \ensuremath{\displaystyle{\sage{vert2med}}}, перпендикуляр: \ensuremath{\displaystyle{\sage{vert1h}}},

длина перпендикуляра: \ensuremath{\displaystyle{\sage{perp}}}
\end{solution}

\begin{question}
Даны вершины треугольника $ABC$: \[A\sage{A}, B\sage{B}, C\sage{C}\].
Составить уравнения биссектрис внутреннего и внешнего углов при вершине $\sagestr{vertex}$.
\end{question}
\begin{solution}
биссектрисы: \\ 
\ensuremath{\displaystyle{\sage{biss1}=0}} -- биссектриса $\sagestr{bissstr}$ угла

\ensuremath{\displaystyle{\sage{biss2}=0}}
\end{solution}

\begin{question}
Даны вершины треугольника $ABC$: \[A\sage{A}, B\sage{B}, C\sage{C}\].
Найти координаты точки $Q$, симметричной точке $\sagestr{vertex}$  относительно прямой $\sagestr{vertex2}\sagestr{vertex3}$.
\end{question}
\begin{solution}
симметричная точка: \ensuremath{\displaystyle{\sage{pp}}}
\end{solution}