\begin{question}
	Разложить вектор $\bm x$ по векторам $\bm p, \bm q, \bm r$ : $\bm x = \alpha\bm p + \beta\bm q + \gamma\bm r$, если
	$\bm{x}=\sage{x11}$, $\bm{p}=\sage{p11}$, $\bm{q}=\sage{q11}$, $\bm{r}=\sage{r11}$.
\end{question}
\begin{solution}
	\ensuremath{\alpha = \sage{coord11[0]}, \beta = \sage{coord11[1]}, \gamma = \sage{coord11[2]}}
\end{solution}
\begin{question}
Найти периметр и площадь треугольника, построенного на векторах $\bm a$ и $\bm b$, если $\bm a=\sage{a12print}$, $\bm b=\sage{b12print}$, а векторы $\bm u$ и $\bm v$ имеют координаты $\bm u=\sage{u12}$, $\bm v=\sage{v12}$.
\end{question}
\begin{solution}
$S=\sage{S12}$, $P=\sage{P12}$
\end{solution}

\begin{question}
Найти угол между диагоналями треугольника, построенного на векторах $\bm a$ и $\bm b$, если $\bm a=\sage{a13print}$, $\bm b=\sage{b13print}$, а про векторы $\bm m$ и $\bm n$ известно, что $|\bm m|=\sage{normm13}$, $|\bm n|=\sage{normn13}$, $\angle(\bm m, \bm n)=\sage{ang13}^\circ$.
\end{question}
\begin{solution}
$\cos\angle(\bm d_1, \bm d_2)=\sage{d1d2ang13}$
\end{solution}
\begin{question}
Даны точки $A\sage{A14}$, $B\sage{B14}$, $C\sage{C14}$. Найти длину вектора $\bm p=\sage{c14} [\sagestr{vec114}, \sagestr{vec214}]$ и его направляющие косинусы.
\end{question}
\begin{solution}
$|\bm p|=\sage{p14norm}$, $\bm e_p=\sage{p14cos}$
\end{solution}

\begin{question}
	Проверить, компланарны ли векторы $\bm a$, $\bm b$ и $\bm c$, если 
	$\bm{a}=\sage{a15}$, $\bm{b}=\sage{b15}$, $\bm{c}=\sage{c15}$.
\end{question}
\begin{solution}
	Векторы \sagestr{compl15}, \ensuremath{\bm{abc} = \sage{abc15}}
\end{solution}
\begin{comment}
\begin{question}
Даны векторы $\bm a$ и $\bm b$. 
\[\bm{a}=\sage{a1}, \bm{b}=\sage{b1}\] 
Найдите вектор $\bm c$, перпендикулярный векторам $\bm a$ и $\bm b$, если проекция вектора $\bm c$ на ось \textit{\sagestr{os}} равна $\sage{prc}$.
\end{question}	
\begin{solution}
$\bm c = \sage{s1[0]}$  
\end{solution}



\begin{question}
Вычислить площадь параллелограмма, построенного на векторах $\bm u$ и $\bm v$, если:
$\bm{u}=\sage{u}, \bm{v}=\sage{v}$, при этом о векторах $\bm a$ и $\bm b$ известно, что: 
$|\bm a|=\sage{normA}, |\bm b| = \sage{normB}$,
$\angle\left(\bm a, \bm b\right)=\sage{ang}^\circ$.
\end{question}
\begin{solution}
\ensuremath{\displaystyle{S = \sage{latex(Sparal)}}}
\end{solution}

\begin{question}
Вычислить угол между диагоналями параллелограмма, построенного на векторах $\bm u$ и $\bm v$, если:
$\bm{u}=\sage{u}, \bm{v}=\sage{v}$, при этом о векторах $\bm a$ и $\bm b$ известно, что: 
$|\bm a|=\sage{normA}, |\bm b| = \sage{normB}$,
$\angle\left(\bm a, \bm b\right)=\sage{ang}^\circ$.
\end{question}
\begin{solution}
\ensuremath{\displaystyle{\cos\angle\left(\bm{d_1}, \bm{d_2}\right)=\sage{angd1d2}}}
\end{solution}



\begin{question}
Вычислить объем тетраэдра с вершинами в точках $A_1, A_2, A_3, A_4$ и его высоту, опущенную из вершины $A_4$ на грань $A_1 A_2 A_3$:
\[A_1\sage{A1}, A_2\sage{A2}, A_3\sage{A3}, A_4\sage{A4}\]
\end{question}
\begin{solution}
Объем тетраэдра \ensuremath{\displaystyle{V = \sage{Vtetr}}}, длина высоты \ensuremath{\displaystyle{h = \sage{htetr}}}
\end{solution}
\end{comment}

