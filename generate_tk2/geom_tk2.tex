\documentclass[a4paper, 12pt]{article}
\usepackage{cmap}					% поиск в PDF
\usepackage{mathtext} 				% русские буквы в формулах
\usepackage[T2A]{fontenc}			% кодировка
\usepackage[utf8]{inputenc}			% кодировка исходного текста
\usepackage[english,russian]{babel}	% локализация и переносы
\usepackage{indentfirst}
\frenchspacing
\usepackage{geometry}
\geometry{left=3cm, right=1.5cm, top=2cm, bottom=1.5cm}
\usepackage{ifthen}
\usepackage{sagetex}
\usepackage{soul}
%\newcounter{zno} % счетчик номеров задач
%%% Дополнительная работа с математикой
\usepackage{amsmath,amsfonts,amssymb,amsthm,mathtools,bm} %
\usepackage{icomma} % "Умная" запятая: $0,2$ --- число, $0, 2$ --- перечисление
\usepackage{etoolbox}
\usepackage{enumitem}
\usepackage{lastpage}
\usepackage{fancyhdr} % Колонтитулы
\pagestyle{fancy}
\renewcommand{\headrulewidth}{0.5pt}  % Толщина линейки, отчеркивающей верхний колонтитул
\renewcommand{\footrulewidth}{0.5pt}
\lfoot{\textit{В.~Н. Кожухова}}
\rfoot{%Иванов Иван Иванович
	\input{\jobname.txt}%
	}
%\rhead{Верхний правый}
\chead{\textbf{ДИТИ НИЯУ МИФИ}, осенний семестр 2017-2018 уч. года}
%\lhead{Верхний левый}
%\cfoot{c. \thepage \; из \pageref{LastPage}} % По умолчанию здесь номер страницы




\makeatletter
\AddEnumerateCounter{\asbuk}{\russian@alph}{щ} % русские буквы в списках
\makeatother

%\renewcommand{\thesection}{Текущий контроль %\arabic{section}.} % Текущий контроль №. Название

%\newcommand{\zadacha}[2]{%
%	\refstepcounter{zno}\label{#1}
%	\textbf{Задача \arabic{zno}}. #2%
%} %Задача №. Текст. На номер можно ссылаться \ref{label}

%\def\teachers{}
%\newcommand{\mycomment}[1]{\ifdef{\teachers}{#1}{\hphantom{#1}%}}
% есть def = показать ответы	

\usepackage{exsheets}
\DeclareTranslation{Russian}{exsheets-exercise-name}{Задача}
\DeclareTranslation{Russian}{exsheets-question-name}{Вопрос}
\DeclareTranslation{Russian}{exsheets-solution-name}{Ответ}
\SetupExSheets[points]{name=б.}

%\def\teachers{}
\newcommand{\mycomment}{\ifdef{\teachers}{\printsolutions}{}}


\begin{document}
\thispagestyle{empty}
%\maketitle

\vspace{-10ex}
	
\begin{center}
\Huge{Аналитическая геометрия}\\
\vspace{-3ex}
\section*{Текущий контроль 2}
\end{center}
\vspace{-2ex}	
\textit{Стоимость: \pointssum* б.}
\begin{flushleft}
Направление подготовки: 09.03.02 Информационные системы и технологии %03.03.02 Физика %14.03.02 Ядерные физика и технологии  

ФИО студента: %Иванов Иван Иванович
 \input{\jobname.txt}

Преподаватель: \textit{В.",Н. Кожухова}	
\end{flushleft}	

\subsection*{Простейшие задачи аналитической геометрии. \\ Полярные координаты}
\begin{sagesilent}
# 1) Вычислить площадь треугольника и найти длину высоты из вершины ??.
A = vector([randint(-7, 7) for i in range(2)])
B = vector([randint(-4, 4) for i in range(2)])
C = vector([randint(-10, 10) for i in range(2)])
AB = B-A
AC = C-A
BC = C-B
crossm = matrix([AB, AC])
while det(crossm)==0 : 
      C = vector([randint(-10, 10) for i in range(2)])
      AC = C-A
      BC = C-B
      crossm = matrix([AB, AC])
SABC= abs(det(crossm)/2)
ch1 = randint(1,3)
if ch1 == 1 :
   vertex = r"A"
   hABC = 2*SABC/BC.norm()
else :
   if ch1 == 2 :
      vertex = r"B"
      hABC = 2*SABC/AC.norm()      
   else :
      vertex = r"C"
      hABC = 2*SABC/AB.norm() 
Gtriag=points([A,B,C], size=50, color='black', figsize=4)+line((A,B), color='black', figsize=4)+line((A,C), color='black', figsize=4)+line((B,C), color='black', figsize=4)
# 2) Центр масс треугольника и углы
angA = arccos(AB.dot_product(AC)/(AB.norm()*AC.norm())) 
angB = arccos((-AB).dot_product(BC)/(AB.norm()*BC.norm()))
angC = arccos((-AC).dot_product(-BC)/(AC.norm()*BC.norm()))
angres = [] 
if (angA==angB) or (angA==angC) or (angB==angC):
   angres.append(r"равнобедренный")
else :
   angres.append(r"не равнобедренный") 
if (angA==pi/2) or (angB==pi/2) or (angC==pi/2):
   angres.append(r"прямоугольный")
else :
   angres.append(r"не прямоугольный") 
# 3) Длина биссектрисы при вершине ...
if ch1 == 1 : # вершина A
   lambd = AB.norm()/AC.norm()
   D = vector([(B[0]+lambd*C[0])/(1+lambd), (B[1]+lambd*C[1])/(1+lambd)])
   AD = D-A
   biss = AD.norm()
else :
   if ch1 == 2 : # вершина B
      lambd = BC.norm()/AB.norm()
      D = vector([(C[0]+lambd*A[0])/(1+lambd), (C[1]+lambd*A[1])/(1+lambd)])
      BD = D-B
      biss = BD.norm()
   else : # вершина C
      lambd = AC.norm()/BC.norm()
      D = vector([(A[0]+lambd*B[0])/(1+lambd), (A[1]+lambd*B[1])/(1+lambd)])
      CD = D-C
      biss = CD.norm()
# 4) Перенос  и поворот системы координат
S = FiniteEnumeratedSet([pi/6, pi/4, pi/3, pi/2, 2*pi/3, 3*pi/4, 5*pi/6, 7*pi/6, 5*pi/4, 4*pi/3, 3*pi/2,5*pi/3,7*pi/4,11*pi/6, arctan(3/4), arctan(5/12), arctan(4/3), arctan(12/5), arctan(8/15), arctan(15/8), arctan(7/24), arctan(24/7)])
ang = S.random_element()
rotm = Matrix([[cos(ang), -sin(ang)],[sin(ang),cos(ang)]])
if ch1 == 1 :
   An = vector([0,0])
   Bn = rotm.inverse()*AB 
   Cn = rotm.inverse()*AC 
else :
   if ch1 == 2 :
      Bn = vector([0,0])
      An = rotm.inverse()*(-AB) 
      Cn = rotm.inverse()*BC 
   else :
      Cn = vector([0,0])
      An = rotm.inverse()*(-AC) 
      Bn = rotm.inverse()*(-BC) 
# Точки в ПСК и ДПСК
def Polar_to_Cart(polar_v) :
    return vector([polar_v[0]*cos(polar_v[1]), polar_v[0]*sin(polar_v[1])])
def Cart_to_Polar(cart_v):
    return vector([sqrt(cart_v[0]^2 + cart_v[1]^2), atan2(cart_v[1],cart_v[0])])
M1 = vector([randint(-7, 7) for i in range(2)])
M2 = vector([randint(-4, 4) for i in range(2)])
M3 = vector([randint(-10, 10) for i in range(2)])
S1 = FiniteEnumeratedSet([pi/6, pi/4, pi/3, pi/2, 2*pi/3, 3*pi/4, 5*pi/6, 7*pi/6, 5*pi/4, 4*pi/3, 3*pi/2,5*pi/3,7*pi/4,11*pi/6,pi,-pi/6, -pi/4, -pi/3, -pi/2, -2*pi/3, -3*pi/4, -5*pi/6, -7*pi/6, -5*pi/4, -4*pi/3, -3*pi/2,-5*pi/3,-7*pi/4,-11*pi/6,-pi])
M4 = vector([randint(1, 10),S1.random_element()])
M5 = vector([randint(1, 10),S1.random_element()])
M6 = vector([randint(1, 10),S1.random_element()])
M4C=Polar_to_Cart(M4)
M5C=Polar_to_Cart(M5)
M6C=Polar_to_Cart(M6)
P2 = {'M1':M1,'M2':M2,'M3':M3,'M4':M4C,'M5':M5C,'M6':M6C}
ymin=min(M1[1],M2[1],M3[1],M4C[1],M5C[1],M6C[1])
ymax=max(M1[1],M2[1],M3[1],M4C[1],M5C[1],M6C[1])
if sign(ymin) == sign(ymax):
   if sign(ymin)==-1:
        ymax=abs(ymax)
   else :
        if sign(ymin)==0:
           ymin=-5
           ymax=5
        else:
           ymin=-ymin   
xmin=min(M1[0],M2[0],M3[0],M4C[0],M5C[0],M6C[0])
xmax=max(M1[0],M2[0],M3[0],M4C[0],M5C[0],M6C[0])
if sign(xmin) == sign(xmax):
  if sign(xmin)==-1:
     xmax=abs(xmax)
  else :
     if sign(xmin)==0:
        xmin=-5
        xmax=5
     else:
        xmin=-xmin
G1 = points(P2.values(), size = 50, color = 'black', aspect_ratio=1, ymin=1.5*ymin, ymax=1.5*ymax, figsize=4, xmin=1.5*xmin,xmax=1.5*xmax)
i=0
for p in P2.keys():
    G1 += text('  %s'%p,P2.values()[i],horizontal_alignment='left',color='black')
    i=i+1
var('x,y,xc,yc,a,b')
eleqn=((x-xc)^2/a^2+(y-yc)^2/b^2-1==0)# уравнение эллипса в канон виде
# первое задание - окружность в центром (0,0)
a=sqrt(randint(1,10)^2/randint(1,10)^2)
b=a
xc=0
yc=0
el2=eleqn(xc=xc,yc=yc,a=a,b=b).expand()
el3=el2.multiply_both_sides(el2.left().denominator()) # конечное уравнение окружности
# окружность в ПСК
var('rho, varphi')
el4=el3.subs(x=rho*cos(varphi), y=rho*sin(varphi)).simplify_trig()
# график окружности
G2=implicit_plot(el3, (x,-1.5*a,1.5*a),(y,-1.5*a,1.5*a), axes='true', figsize=4, color = 'black')
# второе задание - окружность со смещенным центром
a=sqrt(randint(1,10)^2/randint(1,10)^2)# положительное число
ch2=randint(1,2)
if ch2 ==1 :
   a=a
else :
   a=-a # можно и отрицательное
b=a
ch3 = randint(1,2) # сдвиг по X или по Y
if ch3 ==1 :
   xc=a
   yc=0
else :
   xc=0
   yc=a
el5=eleqn(xc=xc,yc=yc,a=a,b=b).expand()
el6=el5.multiply_both_sides(el5.left().denominator()).simplify() # конечное уравнение окружности
# окружность в ПСК
el7=el6.subs(x=rho*cos(varphi), y=rho*sin(varphi)).simplify_trig()
el8=el7.divide_both_sides(rho).expand().solve(rho)[0]
# график окружности
if xc==0:
   if yc<0:
      ymax = abs(a/2)
      ymin = 2*a
      xmin=a
      xmax=-a
   else:
      ymin = -a/2
      ymax = 2*a
      xmin=-a
      xmax=a
else:
   if xc<0:
      xmax = abs(a/2)
      xmin = 2*a
      ymin=a
      ymax=-a
   else:
      xmin = -a/2
      xmax = 2*a
      ymin=-a
      ymax=a
G3=implicit_plot(el6, (x,1.5*xmin,1.5*xmax),(y,1.5*ymin,1.5*ymax), axes='true', figsize=4, color = 'black')
G3+=point((xc,yc), size=25, color='black')
G3 += text('  (%s,%s)'%(xc,yc),(xc+xc/3,yc-yc/3),horizontal_alignment='left',color='black')
# лемнискаты
a=sqrt(randint(1,10)^2/randint(1,10)^2)
ch4=randint(1,2)
if ch4 == 1:
   lemeqn = (x^2+y^2)^2==2*a^2*(x^2-y^2)
else:
   lemeqn = (x^2+y^2)^2==4*a^2*x*y
lem2=lemeqn.subs(x=rho*cos(varphi), y=rho*sin(varphi)).simplify_trig().divide_both_sides(rho^2).expand().reduce_trig()
G4=implicit_plot(lemeqn, (x,-1.5*a,1.5*a),(y,-1.5*a,1.5*a), axes='true', figsize=4, color = 'black')
# спирали
a=sqrt(randint(1,10)^2/randint(1,10)^2)
ch5=randint(1,2)
if ch5 ==1 :
   a=a
else :
   a=-a # можно и отрицательное
spireqn=rho==a*varphi
spi2=(spireqn.subs(rho=sqrt(x^2+y^2), varphi=arctan(y/x)))^2
G5=polar_plot(a*x, (x,0,2*pi), color='black', axes='true', figsize=3)
# кардиоида
a=randint(1,10)
ch6=randint(1,4)
if ch6 ==1 :
   cardeqn = rho==a*(1-sin(varphi))
else :
   if ch6==2:
      cardeqn = rho==a*(1+sin(varphi))
   else:
      if ch6==3:
         cardeqn = rho==a*(1-cos(varphi))
      else:
         cardeqn = rho==a*(1+cos(varphi))
card2=cardeqn.subs(rho=sqrt(x^2+y^2), varphi=arctan(y/x)).factor().canonicalize_radical().multiply_both_sides(sqrt(x^2 + y^2))
if card2.right().coefficient(x,1)==0:
     card3=card2.subtract_from_both_sides(card2.right().coefficient(y,1)*y)
else:
     card3=card2.subtract_from_both_sides(card2.right().coefficient(x,1)*x)
card3=card3^2
G6=polar_plot(cardeqn.right(), (varphi,0,2*pi), color='black', axes='true', figsize=3)
# полярная роза
Spolar=FiniteEnumeratedSet([5/2,-5/2,3/2,-3/2,7/2,-7/2,2,3,4,5,6,7,-2,-3,-4,-5,-6,-7,-2/3,2/3])
a=Spolar.random_element()
b=randint(2,4)
ch7=randint(1,2)
if ch7==1:
   roseeqn=rho==a*cos(b*varphi)
else:
   roseeqn=rho==a*sin(b*varphi)
ros2=((roseeqn.subs(rho=sqrt(x^2+y^2), varphi=arctan(y/x)).simplify_trig().canonicalize_radical())^2).factor()
ros3=ros2.multiply_both_sides(ros2.right().denominator())
G7=polar_plot(roseeqn.right(), (varphi,0,2*pi), color='black', axes='true', figsize=3)
\end{sagesilent}

\vspace{2ex}	
\begin{question}{1}\label{ztriag}
	Даны вершины треугольника $ABC$: \[A\sage{A}, B\sage{B}, C\sage{C}.\] 
\begin{enumerate}
\item Вычислить площадь треугольника и найти длину высоты $\sagestr{vertex}H$.
\item Определить аналитически:
\begin{enumerate}[label=\asbuk*),ref=\asbuk*]
	\item центр масс треугольника $ABC$, считая, что в его вершинах находятся равные массы;		
	\item координаты векторов, приложенных к вершинам треугольника $ABC$  и совпадающих с его медианами;
	\item есть ли тупой угол среди внутренних углов треугольника $ABC$;
	\item является ли треугольник $ABC$ равнобедренным, прямоугольным?
\end{enumerate}	
\end{enumerate}	
\end{question}
\begin{solution}
\sageplot{Gtriag}
	
\ensuremath{
	\displaystyle{S_{ABC}=\sage{SABC}, h=\sage{hABC}}           
}

\ensuremath{\begin{aligned}
		&\text{Центр масс }O\sage{(A+B+C)/3}, \\
		&\text{медианы } AM=\sage{(B+C)/2-A}, BM=\sage{(A+C)/2-B}, CM=\sage{(B+A)/2-C} \\
		&\angle A = \sage{angA} \approx \sage{round(angA*180/pi)}^\circ\\
		&\angle B = \sage{angB} \approx \sage{round(angB*180/pi)}^\circ\\
		&\angle C = \sage{angC} \approx \sage{round(angC*180/pi)}^\circ\\
	\end{aligned}}
	
	Треугольник \sagestr{angres[0]}, \sagestr{angres[1]}.
\end{solution}

\begin{question}{1}
	По данным задачи \ref{ztriag}:
\begin{enumerate}
\item Вычислить длину биссектрисы $\sagestr{vertex}D$ треугольника $ABC$. 

\textit{Примечание}. Координаты точки $D$ необходимо вычислить \textbf{точно}, даже при наличии сложных радикалов. При этом длину биссектрисы в таком случае нужно вычислить приближенно, не забыв записать точный ответ без преобразований.

Пример: если для треугольника с координатами вершин $A(3, 5), B(-4, 4), C(-7, -7)$ требуется вычислить длину биссектрисы, проведенной из вершины $C$, то ответ следует записать так:

$CD = \sqrt{{\left(\dfrac{4 \, \sqrt{130} \sqrt{61} + 325}{\sqrt{130} \sqrt{61} + 65} + 7\right)}^{2} + {\left(-\dfrac{4 \, \sqrt{130} \sqrt{61} - 195}{\sqrt{130} \sqrt{61} + 65} + 7\right)}^{2}}
\approx 12.9$
\item Найти координаты вершин треугольника $ABC$ в новой системе координат, считая что начало координат перенесено в точку $\sagestr{vertex}$, а оси повернуты на угол \ensuremath{\displaystyle{\alpha=\sage{ang}}}.
\end{enumerate}
\end{question}
\begin{solution}
\ensuremath{
	\displaystyle{\sagestr{vertex}D=\sage{biss}\approx\sage{n(biss, digits=4)}}           
}

\ensuremath{
	\displaystyle{A\sage{An}, B\sage{Bn}, C\sage{Cn}}           
}
\end{solution}

\begin{question}{1}
Заданы координаты точек $M_1, M_2, M_3$ в ДПСК и $M_4, M_5, M_6$ в ПСК. Перевести точки $M_1, M_2, M_3$ в ПСК, $M_4, M_5, M_6$ в ДПСК и построить точки $M_1, M_2, M_3$, $M_4, M_5, M_6$, совместив эти системы координат.
\begin{equation*}
\begin{aligned}
M_1\sage{M1}&,  M_2\sage{M2}, M_3\sage{M3},\\
M_4\sage{M4}&,  M_5\sage{M5}, M_6\sage{M6}
\end{aligned}	 
\end{equation*}
\end{question}
\begin{solution}
\ensuremath{\begin{aligned}
		&M_1\sage{Cart_to_Polar(M1)}, M_2\sage{Cart_to_Polar(M2)},  M_3\sage{Cart_to_Polar(M3)} \\
		&M_4\sage{M4C}, M_5\sage{M5C}, M_6\sage{M6C}
	\end{aligned}}
		
	\sageplot{G1}
\end{solution}

\begin{question}{1}
Даны уравнения кривых в ДПСК. Получить их уравнения в ПСК. Построить кривые в ПСК.
\begin{enumerate}
	\item \ensuremath{\displaystyle{\sage{el3}}}
	\item \ensuremath{\displaystyle{\sage{el6}}}
	\item \ensuremath{\displaystyle{\sage{lemeqn}}}
\end{enumerate}	
\end{question}
\begin{solution}
\ensuremath{\displaystyle{\sage{el4.solve(rho)[1]}, \sage{el8}, \sage{lem2}}}

\sageplot{G2}\sageplot{G3}

\sageplot{G4}
\end{solution}

\begin{question}{1}
Даны уравнения кривых в ПСК. Построить эти кривые в ПСК. Получить их уравнения в ДПСК. 
\begin{enumerate}
	\item \ensuremath{\displaystyle{\sage{spireqn}}}
	\item \ensuremath{\displaystyle{\sage{cardeqn}}}
	\item \ensuremath{\displaystyle{\sage{roseeqn}}}
\end{enumerate}
\end{question}
\begin{solution}
\ensuremath{\displaystyle{	\sage{spi2}, \sage{card3},}}

\ensuremath{\displaystyle{\sage{ros3}}}

\sageplot{G5}\sageplot{G6}	

\sageplot{G7}
\end{solution}

\mycomment

\begin{tabular}{|l|*{\numberofquestions}{c|}c|}\hline
Вопрос &
\ForEachQuestion{\QuestionNumber{#1}\iflastquestion{}{&}} &
Всего \\ \hline
Баллы &
\ForEachQuestion{\GetQuestionProperty{points}{#1}\iflastquestion{}{&}} &
\pointssum* \\ \hline
Набрано &
\ForEachQuestion{\iflastquestion{}{&}} & \\ \hline
\end{tabular}




\end{document}
