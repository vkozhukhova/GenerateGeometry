\documentclass[a4paper, 12pt]{article}
\usepackage{cmap}					% поиск в PDF
\usepackage{mathtext} 				% русские буквы в формулах
\usepackage[T2A]{fontenc}			% кодировка
\usepackage[utf8]{inputenc}			% кодировка исходного текста
\usepackage[english,russian]{babel}	% локализация и переносы
\usepackage{indentfirst}
\frenchspacing
\usepackage{geometry}
\geometry{left=3cm, right=1.5cm, top=2cm, bottom=1.5cm}
\usepackage{ifthen}
\usepackage{sagetex}
\usepackage{soul}
%\newcounter{zno} % счетчик номеров задач
%%% Дополнительная работа с математикой
\usepackage{amsmath,amsfonts,amssymb,amsthm,mathtools,bm} %
\usepackage{icomma} % "Умная" запятая: $0,2$ --- число, $0, 2$ --- перечисление
\usepackage{etoolbox}
\usepackage{enumitem}
\usepackage{lastpage}
\usepackage{fancyhdr} % Колонтитулы
\pagestyle{fancy}
\renewcommand{\headrulewidth}{0.5pt}  % Толщина линейки, отчеркивающей верхний колонтитул
\renewcommand{\footrulewidth}{0.5pt}
\lfoot{\textit{В.~Н. Кожухова}}
\rfoot{%Иванов Иван Иванович
	\input{\jobname.txt}%
	}
%\rhead{Верхний правый}
\chead{\textbf{ДИТИ НИЯУ МИФИ}, осенний семестр 2017-2018 уч. года}
%\lhead{Верхний левый}
%\cfoot{c. \thepage \; из \pageref{LastPage}} % По умолчанию здесь номер страницы

\makeatletter
\AddEnumerateCounter{\asbuk}{\russian@alph}{щ} % русские буквы в списках
\makeatother

\usepackage{exsheets}
\DeclareTranslation{Russian}{exsheets-exercise-name}{Задача}
\DeclareTranslation{Russian}{exsheets-question-name}{Вопрос}
\DeclareTranslation{Russian}{exsheets-solution-name}{Ответ}
\SetupExSheets[points]{name=б.}

%\def\teachers{}
\newcommand{\mycomment}{\ifdef{\teachers}{\printsolutions}{}}

\begin{document}
\thispagestyle{empty}

\vspace{-10ex}
	
\begin{center}
\Huge{Аналитическая геометрия}\\
\vspace{-3ex}
\section*{Текущий контроль 4}
\end{center}
\vspace{-2ex}	
\textit{Стоимость: \pointssum* б.}
\begin{flushleft}
Направление подготовки:  09.03.02 Информационные системы и технологии %03.03.02 Физика %14.03.02 Ядерные физика и технологии 

ФИО студента: %Иванов Иван Иванович
 \input{\jobname.txt}

Преподаватель: \textit{В.",Н. Кожухова}	
\end{flushleft}	

\subsection*{Плоскость и прямая в пространстве}
\begin{sagesilent}
var('x,y,z')
M1 = zero_vector(3)
M2 = M1
M3 = M1
a=M1
b=M1
p1=zero_vector(4)
p2=p1
# если направляющий вектор нулевой, перегенерируем
while (M1==M2) or (M1==M3) or (M2==M3) or (M1==zero_vector(3)) or  (M2==zero_vector(3)) or (M3==zero_vector(3)):
  M1 = vector([randint(-5, 5) for i in range(3)])
  M2 = vector([randint(-3, 3) for i in range(3)])
  M3 = vector([randint(-7, 7) for i in range(3)])
while (a==b) or (a==zero_vector(3)) or  (b==zero_vector(3)) or ((M2-M1)==a):
  a = vector([randint(-5, 4) for i in range(3)])
  b = vector([randint(-4, 5) for i in range(3)])   
# составить уравнение плоскости по точке и двум векторам (3 компланарных вектора)
def Make_Plane(point1,v1,v2):
    D=matrix([[x-point1[0],y-point1[1],z-point1[2]],[v1[0],v1[1],v1[2]],[v2[0],v2[1],v2[2]]])
    eqn = D.det()==0
    return eqn
def Simp_Plane(P):
    a=P.left().coefficient(x,1)
    b=P.left().coefficient(y,1)
    c=P.left().coefficient(z,1)
    d=P.left().coefficient(x,0).coefficient(y,0).coefficient(z,0)
    P1=P.divide_both_sides(gcd((a,b,c,d)))
    return P1
P1=Simp_Plane(Make_Plane(M1,M2-M1,M3-M1))
P2=Simp_Plane(Make_Plane(M1,a,b))
P3=Simp_Plane(Make_Plane(M1,M2-M1,a))
a1=randint(-5, 4)
b1=randint(-4, 5)
c1=randint(-7, 7)
pointP=vector([randint(-5, 5) for i in range(3)])
P=a1*(x-pointP[0])+b1*(y-pointP[1])+c1*(z-pointP[2])==0
P=Simp_Plane(P)
L_p1=x==0
L_p2=L_p1
while (p1==p2) or (p1==zero_vector(4)) or (p2==zero_vector(4)) or (L_p1.left() == L_p2.left()):
  p1 = vector([randint(-5, 5) for i in range(4)])
  p2 = vector([randint(-3, 3) for i in range(4)])
  L_p1=Simp_Plane(p1[0]*x+p1[1]*y+p1[2]*z+p1[3]==0)
  L_p2=Simp_Plane(p2[0]*x+p2[1]*y+p2[2]*z+p2[3]==0)
n=vector([a1,b1,c1])	
a2=P2.left().coefficient(x,1)
b2=P2.left().coefficient(y,1)
c2=P2.left().coefficient(z,1)
d2=P2.left().coefficient(x,0).coefficient(y,0).coefficient(z,0)
n2=vector([a2,b2,c2])
P4=Simp_Plane(Make_Plane(M1,n2,n))
P5=Simp_Plane(a1*(x-M1[0])+b1*(y-M1[1])+c1*(z-M1[2])==0)
str_res=[]
if (P.left().subs(x==M3[0],y==M3[1],z==M3[2])==0):
   str_res.append(r'')
else:
   str_res.append(r'не')
if (P.left().subs(x==M2[0],y==M2[1],z==M2[2])==0):
   str_res.append(r'')
else:
   str_res.append(r'не')
if (P4.left().subs(x==M3[0],y==M3[1],z==M3[2])==0):
   str_res.append('')
else:
   str_res.append('не')
if (P4.left().subs(x==M2[0],y==M2[1],z==M2[2])==0):
   str_res.append('')
else:
   str_res.append('не')
# вектор нормали
a4=P4.left().coefficient(x,1)
b4=P4.left().coefficient(y,1)
c4=P4.left().coefficient(z,1)
d4=P4.left().coefficient(x,0).coefficient(y,0).coefficient(z,0)
d1=P.left().coefficient(x,0).coefficient(y,0).coefficient(z,0)
n4=vector([a4,b4,c4])
# делим на +- длину нормального вектора
if d4>0:
   P4norm=P4.divide_both_sides(-n4.norm())
else:
   P4norm=P4.divide_both_sides(n4.norm())
if d1>0:
   Pnorm=P.divide_both_sides(-n.norm())
else:
   Pnorm=P.divide_both_sides(n.norm())
# отклонения точек M2 и M3 от P и P4
deltaM2P=Pnorm.left().subs(x==M2[0],y==M2[1],z==M2[2])
deltaM2P4=P4norm.left().subs(x==M2[0],y==M2[1],z==M2[2])
deltaM3P=Pnorm.left().subs(x==M3[0],y==M3[1],z==M3[2])
deltaM3P4=P4norm.left().subs(x==M3[0],y==M3[1],z==M3[2])
if str_res==['не','не','не','не']:   
     if (sign(deltaM2P)==-sign(deltaM3P)) and (sign(deltaM2P4)==-sign(deltaM3P4)):
        angM2M3str=r"вертикальном" # вертикальные углы
     else:
        if (sign(deltaM2P)==sign(deltaM3P)) and (sign(deltaM2P4)==sign(deltaM3P4)):
            angM2M3str=r"одном" # один и тот же угол
        else:
            angM2M3str=r"смежном" # смежные углы
else:
     angM2M3str=r"----"
sL=(vector([p1[0],p1[1],p1[2]])).cross_product(vector([p2[0],p2[1],p2[2]]))
ML=solve([L_p1.subs(z==0),L_p2.subs(z==0)],x,y)[0]
M2L=solve([L_p1.subs(z==1),L_p2.subs(z==1)],x,y)[0]
pointL=vector([ML[0].right(), ML[1].right(), 0])
point2L=vector([M2L[0].right(), M2L[1].right(), 1])
var('t')
Lparam=(solve([x-pointL[0]==sL[0]*t,y-pointL[1]==sL[1]*t,z==sL[2]*t],x,y,z))[0]
P1t=P1.subs(x==Lparam[0].right(),y==Lparam[1].right(),z==Lparam[2].right())
tt=solve(P1t,t)[0].right()
xLP1=Lparam[0].subs(t=tt)
yLP1=Lparam[1].subs(t=tt)
zLP1=Lparam[2].subs(t=tt)
# запишем наши объекты через sympy
from sympy import Plane, Point, Point3D, Line3D
Psp=Plane(Point3D(pointP), normal_vector=n)
M1sp=Point3D(M1)
prtemp=Psp.projection(M1sp)
# проекция M1 на P
prM1P=vector([prtemp.x._sage_(),prtemp.y._sage_(),prtemp.z._sage_()])
# наша L через sympy
Lsp=Line3D(pointL, point2L)
prtemp=Lsp.projection(M1sp)
# проекция M1 на L
prM1L=vector([prtemp.x._sage_(),prtemp.y._sage_(),prtemp.z._sage_()])
# расстояние от M2 до P2
P2sp=Plane(Point3D(M1), normal_vector=n2)
dM2P2=P2sp.distance(Point3D(M2))
# плоскость через M1 и L
P6=Simp_Plane(Make_Plane(M1,pointL-M1,sL))
# проекция L1 на P
# уравнение L1 в параметрическом виде
pointL1=vector([M2[0]+a[0],M2[1]+a[1],M2[2]+a[2]])
M2sp=Point3D(M2)
pointL1sp=Point3D(pointL1)
# проекция M2 на P
prtemp=Psp.projection(M2sp)
prM2P=vector([prtemp.x._sage_(),prtemp.y._sage_(),prtemp.z._sage_()])
# проекция pointL1 на P
prtemp=Psp.projection(pointL1sp)
prpointL1P=vector([prtemp.x._sage_(),prtemp.y._sage_(),prtemp.z._sage_()])
prL1Pparam=(solve([x-prM2P[0]==(prpointL1P[0]-prM2P[0])*t,y-prM2P[1]==(prpointL1P[1]-prM2P[1])*t,z-prM2P[2]==(prpointL1P[2]-prM2P[2])*t],x,y,z))[0]
L2sp=Line3D(Point3D(M2), Point3D(M3))
# проверка на компланарность (=, пересекаются или параллельны)
w=Point3D.are_coplanar(Point3D(pointL), Point3D(point2L), Point3D(M2), Point3D(M3))
sL2=M3-M2
sLL2=M2-pointL
if w:
   if Line3D.is_parallel(Lsp,L2sp):
      strLL2=r"параллельны"
      dLL2=Lsp.distance(M2)
   else:
      if Line3D.is_similar(Lsp,L2sp):
         strLL2=r"совпадают"
         dLL2=0
      else:
         strLL2=r"пересекаются"
         dLL2=0
else:
   strLL2=r"скрещиваются"
   dLL2=abs(sLL2.dot_product(sL.cross_product(sL2)))/(sL.cross_product(sL2)).norm()
# уравнения биссектрис угла между плоскостями P и P4
denom=gcd((n.norm(),n4.norm()))
biss3=Pnorm.left()-P4norm.left()
biss4=Pnorm.left()+P4norm.left()
biss3=biss3.canonicalize_radical().mul(denom)
biss4=biss4.canonicalize_radical().mul(denom)
biss3=biss3.expand().collect(x).collect(y).collect(z)
biss4=biss4.expand().collect(x).collect(y).collect(z)
biss3=biss3.mul(biss3.denominator())
biss4=biss4.mul(biss4.denominator())
\end{sagesilent}

\vspace{2ex}	

\begin{question}{1}\label{base}
Заданы координаты точек $M_1, M_2, M_3$ в пространстве, векторы $\bm a$ и $\bm b$, плоскость $P$ и прямая $L$. 
\[M_1\sage{M1}, M_2\sage{M2}, M_3\sage{M3}, \bm a=\sage{a}, \bm b=\sage{b},\]
\[P: \sage{P}, L: \left\{\begin{array}{l}
\sage{L_p1} \\
\sage{L_p2}
\end{array}\right.
\]
\begin{enumerate}
	\item	Составить уравнение плоскости $P_1$, проходящей через точки $M_1, M_2, M_3$.
	\item	Составить уравнение плоскости $P_2$, проходящей через точку $M_1$, параллельно векторам $\bm a$ и $\bm b$.
	\item	Составить уравнение плоскости $P_3$, проходящей через точки $M_1, M_2$, параллельно вектору $\bm a$.
\end{enumerate}
\end{question}
\begin{solution}
$P_1: \sage{P1}$

$P_2: \sage{P2}$

$P_3: \sage{P3}$
\end{solution}

\begin{question}{1}\label{base2}
По данным задачи \ref{base}:
\begin{enumerate}
	\item Составить уравнение плоскости $P_4$, содержащей точку $M_1$  и перпендикулярной плоскостям $P_2$ и $P$.
	\item Составить уравнение плоскости $P_5$, содержащей точку $M_1$  и параллельной плоскости $P$.
	\item Определить, лежат ли точки $M_2, M_3$ на плоскостях $P_4$ и $P$. Если нет, то выяснить лежат ли они в одном, в смежных или в вертикальных двугранных углах, образованных при пересечении плоскостей $P_4$ и $P$.
\end{enumerate}	
\end{question}
\begin{solution}
$P_4: \sage{P4}$

$P_5: \sage{P5}$

$M_3$ $\sagestr{str_res[0]}$ лежит в плоскости $P$

$M_2$ $\sagestr{str_res[1]}$ лежит в плоскости $P$

$M_3$ $\sagestr{str_res[2]}$ лежит в плоскости $P_4$

$M_2$ $\sagestr{str_res[3]}$ лежит в плоскости $P_4$

$M_2$ и $M_3$ находятся в $\sagestr{angM2M3str}$ углу	
\end{solution}

\begin{question}{1}
По данным задачи \ref{base}:
\begin{enumerate}
	\item Написать канонические и параметрические уравнения прямой $L$.
	\item Найти точку пересечения прямой $L$ и плоскости $P_1$.
	\item Найти проекцию точки $M_1$ на прямую $L$.
	\item Найти проекцию точки $M_1$ на плоскость $P$.
\end{enumerate}	
\end{question}
\begin{solution}
	Канонические уравнения: \ensuremath{\displaystyle{\frac{\sage{x-pointL[0]}}{\sage{sL[0]}}=\frac{\sage{y-pointL[1]}}{\sage{sL[1]}}=\frac{z}{\sage{sL[2]}}}}
	
	Параметрические уравнения:
\ensuremath{\displaystyle{\sage{Lparam}}}
	
	Точка пересечения прямой и плоскости: \ensuremath{\displaystyle{\sage{xLP1}, \sage{yLP1}, \sage{zLP1}}}
	
	Проекция точки $M_1$ на плоскость $P$: \ensuremath{\displaystyle{\sage{prM1P}}}
	
	Проекция точки $M_1$ на прямую $L$: \ensuremath{\displaystyle{\sage{prM1L}}}
\end{solution}

\begin{question}{1}
	По данным задачи \ref{base}:
\begin{enumerate}
\item Написать уравнения прямой $L_1$, проходящей через точку $M_2$   параллельно вектору $\bm a$. Доказать, что прямая $L_1$ параллельна   плоскости $P_2$ и найти расстояние между $L_1$ и $P_2$.
\item Написать уравнение плоскости $P_6$, проходящей через точку $M_1$ и прямую $L$. 
\item Найти проекцию прямой $L_1$ на плоскость $P$. Ответ записать в каноническом виде.
\end{enumerate}
\end{question}
\begin{solution}
Прямая $L_1$: \ensuremath{
	\displaystyle{
		\frac{\sage{x-M2[0]}}{\sage{a[0]}}=
		\frac{\sage{y-M2[1]}}{\sage{a[1]}}=
			\frac{\sage{z-M2[2]}}{\sage{a[2]}}
				}}

Расстояние между $L_1$ и $P_2$: \ensuremath{\displaystyle{\sage{dM2P2._sage_()}}}

Уравнение плоскости $P_6$: $\sage{P6}$

Проекция прямой $L_1$ на плоскость $P$: \ensuremath{
	\displaystyle{
		\dfrac{\sage{x-prM2P[0]}}{\sage{prpointL1P[0]-prM2P[0]}}=
		\dfrac{\sage{y-prM2P[1]}}{\sage{prpointL1P[1]-prM2P[1]}}=
		\dfrac{\sage{z-prM2P[2]}}{\sage{prpointL1P[2]-prM2P[2]}}
	}}
	
\ensuremath{
	\displaystyle{\sage{prL1Pparam}}}
\end{solution}
\begin{question}{1}
	По данным задач \ref{base} и \ref{base2}:
	\begin{enumerate}
\item Выяснить взаимное расположение прямой $L$ и прямой $L_2$, проходящей через точки $M_2$ и $M_3$. Найти расстояние между $L$ и $L_2$. 
\item Написать уравнение плоскости $P_7$, делящей пополам тот двугранный угол между плоскостями $P$ и $P_4$, в котором расположена точка $M_2$. 	
	\end{enumerate}
\end{question}
\begin{solution}
Прямые $L$ и $L_2$ \sagestr{strLL2}.

Расстояние между $L$ и $L_2$: \ensuremath{
	\displaystyle{\sage{dLL2}}}

Биссектрисы углов между $P$ и $P_4$: 
 
\ensuremath{\scriptstyle{\sage{biss3}=0}}

\ensuremath{\scriptstyle{\sage{biss4}=0}}
\end{solution}

\mycomment

\begin{tabular}{|l|*{\numberofquestions}{c|}c|}\hline
	Вопрос &
	\ForEachQuestion{\QuestionNumber{#1}\iflastquestion{}{&}} &
	Всего \\ \hline
	Баллы &
	\ForEachQuestion{\GetQuestionProperty{points}{#1}\iflastquestion{}{&}} &
	\pointssum* \\ \hline
	Набрано &
	\ForEachQuestion{\iflastquestion{}{&}} & \\ \hline
\end{tabular}

\end{document}
