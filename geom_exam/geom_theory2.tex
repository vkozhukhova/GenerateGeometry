\begin{question}
Общее уравнение плоскости. Уравнение плоскости, проходящей через 3
точки.
\end{question}
\begin{question}
Уравнение плоскости в отрезках. Нормальное уравнение плоскости.
\end{question}
\begin{question}
Угол между плоскостями. Взаимное расположение плоскостей. Условия
параллельности и перпендикулярности плоскостей.
\end{question}
\begin{question}
Прямая в пространстве. Канонические и параметрические уравнения
прямой.
\end{question}
\begin{question}
Прямая как пересечение двух плоскостей. Приведение к каноническому
виду.
\end{question}
\begin{question}
Расстояние от точки до прямой в пространстве.
\end{question}
\begin{question}
Скрещивающиеся прямые. Условие принадлежности двух прямых одной
плоскости.
\end{question}
\begin{question}
Расстояние между прямыми в пространстве.
\end{question}
\begin{question}
Расстояние между параллельными плоскостями в пространстве.
\end{question}
\begin{question}
Взаимное расположение прямой и плоскости. Угол между прямой и
плоскостью.
\end{question}
\begin{question}
Пересечение прямой и плоскости. Проекция точки на прямую и плоскость.
\end{question}
\begin{question}
Эллипс: каноническое уравнение, свойства.
\end{question}
\begin{question}
Гипербола: каноническое уравнение, свойства.
\end{question}
\begin{question}
Парабола: каноническое уравнение, свойства.
\end{question}
\begin{question}
Приведение к каноническому виду общего уравнения кривой 2-го
порядка путём переноса начала координат.
\end{question}
\begin{question}
Приведение к каноническому виду общего уравнения кривой 2-го
порядка путём поворота осей координат.
\end{question}

\begin{question}
Инварианты. Классификация кривых 2-го порядка с помощью
инвариантов.
\end{question}
\begin{question}
Поверхности вращения и преобразование сжатия. Эллипсоид.
\end{question}
\begin{question}
Гиперболоиды (однополостный, двуполостный).
\end{question}
\begin{question}
Параболоиды (эллиптический, гиперболический).
\end{question}
\begin{question}
Конические поверхности. Конус 2-го порядка.
\end{question}
\begin{question}
Цилиндрические поверхности. Цилиндры 2-го порядка.
\end{question}
\begin{question}
Исследование формы поверхности 2-го порядка методом параллельных
сечений (пример).
\end{question}