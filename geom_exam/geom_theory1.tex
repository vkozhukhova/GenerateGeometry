\begin{question}
Вектор. Линейные операции над векторами и их свойства.
\end{question}
\begin{question}
Проекция вектора на ось. Направляющие косинусы вектора.
\end{question}
\begin{question}
Линейная зависимость системы векторов.
\end{question}
\begin{question}
Основные теоремы о линейной зависимости системы 3-х, 4-х векторов.
\end{question}
\begin{question}
Понятие базиса. Разложение вектора по базису. Ортонормированные
базисы.
\end{question}
\begin{question}
Скалярное произведение и его свойства. Условие ортогональности.
\end{question}
\begin{question}
Векторное произведение и его свойства. Условие коллинеарности.
Геометрический смысл векторного произведения.
\end{question}
\begin{question}
Смешанное произведение и его свойства, условие компланарности
векторов. Геометрический смысл смешанного произведения.
\end{question}
\begin{question}
Декартова система координат (ДПСК). Параллельный перенос, поворот.
\end{question}
\begin{question}
Полярная система координат. Переход от ПСК к ДПСК и обратно.
\end{question}
\begin{question}
Классификация кривых на плоскости. Трансцендентные кривые: примеры.
\end{question}
\begin{question}
Кривые в полярной системе координат: окружность со смещенным
центром, кардиоида. Переход к уравнениям в ДПСК.
\end{question}
\begin{question}
Кривые в полярной системе координат: спираль Архимеда, лемниската
Бернулли, полярная роза. Переход к уравнениям в ДПСК.
\end{question}

\begin{question}
Прямая на плоскости. Общее уравнение (вывод).
\end{question}
\begin{question}
Каноническое и параметрические уравнения прямой.
\end{question}
\begin{question}
Уравнение прямой с угловым коэффициентом. Угол между прямыми.
\end{question}
\begin{question}
Условие параллельности и перпендикулярности прямых.
\end{question}
\begin{question}
Уравнение прямой, проходящей через 2 точки. Уравнение прямой в
отрезках.
\end{question}
\begin{question}
Нормальное уравнение прямой. Расстояние от точки до прямой.
\end{question}
\begin{question}
Неполные уравнения прямой. Некоторые типовые задачи на прямую на
плоскости: проведение прямой под некоторым углом к данной;
построение уравнений биссектрис угла между пересекающимися
прямыми; условие пересечения прямой заданного отрезка.
\end{question}
\begin{question}
Некоторые типовые задачи на прямую на плоскости: расположение начала
координат относительно пересекающихся прямых; расположение начала
координат и заданной точки относительно пересекающихся прямых.
\end{question}
\begin{question}
Пучок прямых на плоскости. Основные задачи, решаемые с помощью
уравнения пучка прямых на плоскости.
\end{question}