% !TeX program = xelatex
\documentclass[a4paper, 12pt,fleqn]{article}
\usepackage{amsmath,amsfonts,amssymb,amsthm,graphics, setspace,enumitem, unicode-math,mathtools}
\usepackage[english, russian]{babel}
\usepackage{fontspec}
%%% Дополнительная работа с математикой
\usepackage{icomma} % "Умная" запятая: $0,2$ --- число, $0, 2$ --- перечисление
\usepackage{etoolbox}
\usepackage{enumitem}
\usepackage{verbatim}
\setmainfont{Times New Roman}
\setmathfont{TeX Gyre Termes Math}
\usepackage{ifthen}
\usepackage{soul}
\usepackage{sagetex}
%\usepackage{lscape}
\usepackage{geometry}
\geometry{left=10mm, right=5mm, top=10mm, bottom=5mm}
\setlength\mathindent{0pt}
\baselineskip=0pt

\def\examno{1}
\def\firstseed{1000}
\def\secondseed{5000}

\newcommand\invisiblesection[1]{%
	\refstepcounter{section}%
	\addcontentsline{toc}{section}{\protect\numberline{\thesection}#1}%
	\sectionmark{#1}}

\pagestyle{empty}

%%% Работа с картинками
\usepackage{graphicx}  % Для вставки рисунков
\graphicspath{{images/}{images2/}}  % папки с картинками
\setlength\fboxsep{3pt} % Отступ рамки \fbox{} от рисунка
\setlength\fboxrule{1pt} % Толщина линий рамки \fbox{}
\usepackage{wrapfig} % Обтекание рисунков и таблиц текстом
\usepackage[nomessages]{fp}

\newcommand{\shapka} {
\begin{center}
		\fontspec{Times New Roman}
		М\,И\,Н\,И\,С\,Т\,Е\,Р\,С\,Т\,В\,О\; О\,Б\,Р\,А\,З\,О\,В\,А\,Н\,И\,Я \ И \ Н\,А\,У\,К\,И \ Р\,О\,С\,С\,И\,Й\,С\,К\,О\,Й \ Ф\,Е\,Д\,Е\,Р\,А\,Ц\,И\,И \\
			\begin{tiny} ФЕДЕРАЛЬНОЕ ГОСУДАРСТВЕННОЕ АВТОНОМНОЕ ОБРАЗОВАТЕЛЬНОЕ УЧРЕЖДЕНИЕ ВЫСШЕГО ОБРАЗОВАНИЯ 	\end{tiny}\\
			\begin{small}  «Национальный исследовательский ядерный университет «МИФИ» \end{small} \\
			\fontspec{Book Antiqua}
		\begin{large}	\textbf{Димитровградский инженерно-технологический институт} \end{large} – \\
			\begin{footnotesize} филиал федерального государственного автономного  
				образовательного учреждения высшего образования \\
				«Национальный исследовательский ядерный университет «МИФИ» \end{footnotesize}\\
			\textbf{(ДИТИ НИЯУ МИФИ)}\par
\fontspec{Times New Roman}
%\vspace{\baselineskip}
		\begin{tabular}{p{0.5\linewidth}  l }
			\textbf{Кафедра}: Высшей математики & Утверждено на заседании кафедры  \\ 
			\textbf{Факультет}: Физико-технический & Протокол от «\rule{0.5 cm}{0.5 pt}» \rule{2 cm}{0.5 pt} 20\rule{0.4 cm}{0.5 pt}г. №\rule{0.7 cm}{0.5 pt} \\  
			\textbf{Направление}:  03.03.02 Физика & Зав. кафедрой \rule{2 cm}{0.5 pt} Т.И. Романовская \\
			\textbf{Дисциплина}: Аналитическая геометрия & \\   
			\textbf{Форма обучения}: Очная &  \textbf{Курс  1 }   
		\end{tabular}	
\end{center}}
\newenvironment{bilet}[1]{%
%	\addtocounter{examno}{1}%
\shapka %
\centerline{\textbf{ЭКЗАМЕНАЦИОННЫЙ БИЛЕТ № #1}} 
\invisiblesection{#1}%\pgfmathprintnumber\pgfmathresult
}{\flushleft Составил преподаватель: \rule{3 cm}{0 pt} В.Н. Кожухова}	

% команда повтора текста
\begin{comment}\makeatletter
\newcommand\myrepeat[3]{%
	% #1 is the number of repetitions
	% #2 is the code to repeat
	% #3 is the code to put in the middle
	\expandafter\myrepeat@aux\expandafter{\romannumeral\number\number#1 000}{#2}{#3}%
}
\newcommand{\myrepeat@aux}[3]{\myrepeat@auxi{#2}{#3}#1;;}
\def\myrepeat@auxi#1#2#3#4{%
	\ifx#3;%
	\expandafter\@gobble % recursion has ended
	\else
	\expandafter\@firstofone % still one m to swallow
	\fi
	{\myrepeat@auxii{#1}{#2}{#4}}%
}
\def\myrepeat@auxii#1#2#3{%
	#1\ifx#3;\else#2\fi
	\myrepeat@auxi{#1}{#2}#3% restart the recursion
}
\makeatletter
\end{comment}
\usepackage{exsheets}
\DeclareTranslation{Russian}{exsheets-exercise-name}{}
\DeclareTranslation{Russian}{exsheets-question-name}{Вопрос}
\DeclareTranslation{Russian}{exsheets-solution-name}{Ответ}
\SetupExSheets{counter-within=section}
\SetupExSheets{headings=runin}

\begin{document}
	\begin{sagesilent}
		# Разложение вектора по векторам
p11=zero_vector(3)
q11=p11
r11=p11
mpqr11 = matrix([p11, q11, r11])
while p11==zero_vector(3) or q11==zero_vector(3) or r11==zero_vector(3) or det(mpqr11)==0:
	 p11 = vector([randint(-7, 7) for i in range(3)])
	 q11 = vector([randint(-3, 3) for i in range(3)])
	 r11 = vector([randint(-10, 10) for i in range(3)])
	 mpqr11 = matrix([p11, q11, r11])
coord11 = vector([0,0,0])
while coord11 == vector([0,0,0]) : 
	 coord11 = vector([randint(-4, 4) for i in range(3)])
	 x11 = coord11[0]*p11 + coord11[1]*q11 + coord11[2]*r11
# Найти периметр и площадь треугольника на векторах a и b
var('u,v,m,n')
u12=zero_vector(3) 
v12=u12
while u12==zero_vector(3) or v12==zero_vector(3) or u12.cross_product(v12)==zero_vector(3):
	 u12 = vector([randint(-3, 4) for i in range(3)])
	 v12 = vector([randint(-4, 3) for i in range(3)])
basisuv12 = vector([u,v])
acoord12=zero_vector(2)
bcoord12=zero_vector(2)
mab12 = matrix([acoord12, bcoord12])
while acoord12[0]==0 or acoord12[1]==0 or bcoord12[0]==0 or bcoord12[1]==0 or det(mab12)==0:
	 acoord12 = vector([randint(-3, 4) for i in range(2)])
	 bcoord12 = vector([randint(-4, 3) for i in range(2)])
	 mab12 = matrix([acoord12, bcoord12])
a12print=basisuv12.dot_product(acoord12)
b12print=basisuv12.dot_product(bcoord12)
a12=acoord12[0]*u12+acoord12[1]*v12
b12=bcoord12[0]*u12+bcoord12[1]*v12
S12=(a12.cross_product(b12)).norm()/2
P12=a12.norm()+b12.norm()+(b12-a12).norm()
	\end{sagesilent}
%\begin{landscape}
%\fontspec{Times New Roman}
\setstretch{0.95}
\abovedisplayskip = 0pt
\belowdisplayskip = 0pt
\abovedisplayshortskip = 0pt
\belowdisplayshortskip= 0pt
\FPeval{\result}{clip(2*\examno-1)}
	\begin{bilet}{\result} \pgfmathsetseed{\firstseed}
	\includequestions[random=1]{geom_theory1.tex}	
	\includequestions[random=1]{geom_theory2.tex}
    \includequestions[random=1]{geom_vectors.tex}
\end{bilet}

\FPeval{\result}{clip(2*\examno)}
	\begin{bilet}{\result} \pgfmathsetseed{\secondseed}	
		
	\includequestions[random=1]{geom_theory1.tex}
	\includequestions[random=1]{geom_theory2.tex}	
    \includequestions[random=1]{geom_vectors.tex}
	\end{bilet}

%\end{landscape}

\end{document}	