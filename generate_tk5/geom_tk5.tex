\documentclass[a4paper, 12pt]{article}
\usepackage{cmap}					% поиск в PDF
\usepackage{mathtext} 				% русские буквы в формулах
\usepackage[T2A]{fontenc}			% кодировка
\usepackage[utf8]{inputenc}			% кодировка исходного текста
\usepackage[english,russian]{babel}	% локализация и переносы
\usepackage{indentfirst}
\frenchspacing
\usepackage{geometry}
\geometry{left=3cm, right=1.5cm, top=2cm, bottom=1.5cm}
\usepackage{ifthen}
\usepackage{sagetex}
\usepackage{soul}
%\newcounter{zno} % счетчик номеров задач
%%% Дополнительная работа с математикой
\usepackage{amsmath,amsfonts,amssymb,amsthm,mathtools,bm} %
\usepackage{icomma} % "Умная" запятая: $0,2$ --- число, $0, 2$ --- перечисление
\usepackage{etoolbox}
\usepackage{enumitem}
\usepackage{lastpage}
\usepackage{fancyhdr} % Колонтитулы
\pagestyle{fancy}
\renewcommand{\headrulewidth}{0.5pt}  % Толщина линейки, отчеркивающей верхний колонтитул
\renewcommand{\footrulewidth}{0.5pt}
\lfoot{\textit{В.~Н. Кожухова}}
\rfoot{%Иванов Иван Иванович
	\input{\jobname.txt}%
	}
%\rhead{Верхний правый}
\chead{\textbf{ДИТИ НИЯУ МИФИ}, осенний семестр 2017-2018 уч. года}
%\lhead{Верхний левый}
%\cfoot{c. \thepage \; из \pageref{LastPage}} % По умолчанию здесь номер страницы

\makeatletter
\AddEnumerateCounter{\asbuk}{\russian@alph}{щ} % русские буквы в списках
\makeatother

\usepackage{exsheets}
\DeclareTranslation{Russian}{exsheets-exercise-name}{Задача}
\DeclareTranslation{Russian}{exsheets-question-name}{Вопрос}
\DeclareTranslation{Russian}{exsheets-solution-name}{Ответ}
\SetupExSheets[points]{name=б.}

%\def\teachers{}
\newcommand{\mycomment}{\ifdef{\teachers}{\printsolutions}{}}

\begin{document}
\thispagestyle{empty}

\vspace{-10ex}
	
\begin{center}
\Huge{Аналитическая геометрия}\\
\vspace{-3ex}
\section*{Текущий контроль 5}
\end{center}
\vspace{-2ex}	
\textit{Стоимость: \pointssum* б.}
\begin{flushleft}
Направление подготовки:  09.03.02 Информационные системы и технологии %03.03.02 Физика %14.03.02 Ядерные физика и технологии 

ФИО студента: %Иванов Иван Иванович
 \input{\jobname.txt}

Преподаватель: \textit{В.",Н. Кожухова}	
\end{flushleft}	

\subsection*{Кривые и поверхности второго порядка}
\begin{sagesilent}
# составление уравнение эллипса и параболы
var('x,y,rho,varphi')
# полуоси и фокальный параметр
a = randint(1, 10)
b=a
p=0
curve=randint(1,2)
Os=randint(1, 2)
while a==b or p==0:
  b=randint(1, 10)
  p=randint(-10, 10)
# ось симметрии и полуоси
if Os==1:
  symm=r'Ox'
else:
  symm=r'Oy'
if curve==1:
  txtcurve=r'эллипса'
else:
  txtcurve=r'гиперболы'
if Os==1 and curve==1:
  # эллипс
  answ1=x^2/a^2+y^2/b^2==1
  c=sqrt(abs(a^2-b^2))
  if a>b:
    F1=vector([c,0])
    F2=vector([-c,0])
  else:
    F1=vector([0,c])
    F2=vector([0,-c])    
  # парабола
  answ2=y^2==2*p*x
  F=vector([p/2,0])
else:
  if Os==1 and curve==2:
     # гипербола
     answ1=x^2/a^2-y^2/b^2==1
     c=sqrt(a^2+b^2)
     F1=vector([c,0])
     F2=vector([-c,0])
     # парабола
     answ2=y^2==2*p*x
     F=vector([p/2,0])
  else:
     if Os==2 and curve==1:
        # эллипс
        answ1=x^2/a^2+y^2/b^2==1
        c=sqrt(abs(a^2-b^2))
        if a>b:
          F1=vector([c,0])
          F2=vector([-c,0])
        else:
          F1=vector([0,c])
          F2=vector([0,-c])    
        # парабола
        answ2=x^2==2*p*y
        F=vector([0,p/2])
     else:
        # гипербола
        answ1=x^2/a^2-y^2/b^2==1
        c=sqrt(a^2+b^2)
        F1=vector([c,0])
        F2=vector([-c,0])
        # парабола
        answ2=x^2==2*p*y
        F=vector([0,p/2])
eq1=answ1
eq2=eq1.multiply_both_sides(lcm(a^2,b^2))
eq2=eq2.subs(x==rho*cos(varphi), y==rho*sin(varphi))
eq2=eq2.trig_simplify()    
if eq2.find(cos(varphi))==[]:
   eq2=eq2.subs(sin(varphi)^2==1/2-cos(2*varphi)/2)
else:
   eq2=eq2.subs(cos(varphi)^2==1/2+cos(2*varphi)/2)
eq2=eq2.left().collect(rho^2)==eq2.right()
eq2=eq2.divide_both_sides(eq2.left()).multiply_both_sides(rho^2)
eq3=rho==sqrt(eq2.right()).simplify()
polar1=eq3 
eq1=answ2
eq2=eq1.multiply_both_sides(lcm(a^2,b^2))
eq2=eq2.subs(x==rho*cos(varphi), y==rho*sin(varphi))
eq2=eq2.trig_simplify()    
if eq2.find(cos(varphi))==[]:
  eq2=eq2.subs(sin(varphi)^2==1/2-cos(2*varphi)/2)
else:
  eq2=eq2.subs(cos(varphi)^2==1/2+cos(2*varphi)/2) 
eq2=eq2.divide_both_sides(rho)
eq3=eq2.divide_both_sides(eq2.left()).multiply_both_sides(rho)
polar2=eq3
G1=implicit_plot(answ1, (x,-2*max(a,c),2*max(a,c)), (y,-2*max(a,c),2*max(a,c)), axes='true', figsize=3, color = 'black')
#G1=polar_plot(polar1.right(), (varphi,0,2*pi),color='black', axes='true', #figsize=3)
G2=implicit_plot(answ2, (x,-10,10), (y,-10,10), axes='true', figsize=3, color = 'black')
# составление уравнения по точке и расстоянию до кривой
a1=0
b1=0
while a1==0 or b1==0:
   A=vector([randint(-7, 7) for i in range(2)])
   S=FiniteEnumeratedSet([1/2,1/3,1,2,3,3/2,2/3])
   n=S.random_element()
   ch1=randint(1,2)
   c=randint(-4,7)
   if ch1==1:
        l=x-c==0
   else:
        l=y-c==0
   eq1=sqrt((x-A[0])^2+(y-A[1])^2)==n*(l.left())
   eq2=eq1^2
   eq3=eq2.canonicalize_radical()
   eq4=eq3.left()-eq3.right()==0
   a1=eq4.left().coefficient(x,2)
   a2=eq4.left().coefficient(x,1)
   c1=eq4.left().coefficient(x,0).coefficient(y,0)
   b1=eq4.left().coefficient(y,2)
   b2=eq4.left().coefficient(y,1)
eq5=x^2+a2*x/a1
eq6=y^2+b2*y/b1
var('cx,cy,k1,k2')
from sympy import solve
s1=solve(eq5-((x-cx)^2+k1),[cx,k1])
s2=solve(eq6-((y-cy)^2+k2),[cy,k2])
eq7=a1*(x-s1[0][0])^2+b1*(y-s2[0][0])^2==-a1*s1[0][1]-b1*s2[0][1]-c1
if eq7.rhs()==0:
   answ3=eq7
else:
   answ3=eq7.divide_both_sides(eq7.rhs())
# приведение к каноническому виду
# гиперболический тип
def MakeEquationHypElps(a,b,ch1,mysgn):
    if mysgn==1:
      eq=x^2/a^2+y^2/b^2==ch1
    else:
      eq=x^2/a^2-y^2/b^2==ch1
    return eq
ah=0
bh=0
while ah==bh:
    bh=randint(1, 10)
    ah=randint(1, 10)
ch1h=randint(-1,1)
hyperb=MakeEquationHypElps(ah, bh, ch1h, 2) 
ae=0
be=0
while ae==be:
   be=randint(1, 10)
   ae=randint(1, 10)
ch1e=randint(-1,1)
elps=MakeEquationHypElps(ae, be, ch1e, 1) 
def CanonToGeneralHypElps(eq,a,b):
    x0=randint(-10,10)
    y0=randint(-10,10)
    S = FiniteEnumeratedSet([pi/6, pi/4, pi/3, 2*pi/3, 3*pi/4, 5*pi/6, 7*pi/6, 5*pi/4, 4*pi/3, 5*pi/3,7*pi/4,11*pi/6, arctan(3/4), arctan(5/12), arctan(4/3), arctan(12/5), arctan(8/15), arctan(15/8), arctan(7/24), arctan(24/7)])
    ang = S.random_element()
    eq1=eq.add_to_both_sides(-eq.right()).simplify()
    eq2=eq1.subs(x==x*cos(ang)+y*sin(ang), y==-x*sin(ang)+y*cos(ang))
    eq3=eq2.canonicalize_radical()
    eq4=eq3.multiply_both_sides(eq3.left().denominator())
    eq5=eq4.subs(x==x-x0,y==y-y0).expand()
    return [x0,y0,ang,eq5]
hyp_task=CanonToGeneralHypElps(hyperb,ah,bh)
G3=implicit_plot(hyp_task[3].left(), (x,-50,50), (y,-50,50), axes='true', figsize=3, color = 'red')
elps_task=CanonToGeneralHypElps(elps,ae,be)
G4=implicit_plot(elps_task[3].left(), (x,-50,50), (y,-50,50), axes='true', figsize=3, color = 'blue')
# параболу делаем!
ch1=randint(1,6)
ch2=randint(1,2)
p1=0
while p1==0:
    p1=randint(-10,10)
if ch1<=4:
     if ch2==1:
         parab=y^2==2*p1*x
     else:
         parab=x^2==2*p1*y
else:   
     if ch2==1 and p1>0:
         parab=y^2==p1^2
     elif ch2==1 and p1<0:
         parab=y^2==-p1^2
     elif ch2==2 and p1>0:
         parab=x^2==p1^2
     else:
         parab=x^2==-p1^2     
def CanonToGeneralParab(eq):
    x0=randint(-10,10)
    y0=randint(-10,10)
    S = FiniteEnumeratedSet([pi/6, pi/4, pi/3, 2*pi/3, 3*pi/4, 5*pi/6, 7*pi/6, 5*pi/4, 4*pi/3, 5*pi/3,7*pi/4,11*pi/6, arctan(3/4), arctan(5/12), arctan(4/3), arctan(12/5), arctan(8/15), arctan(15/8), arctan(7/24), arctan(24/7)])
    ang = S.random_element()
    eq1=eq.add_to_both_sides(-eq.right()).simplify()
    eq2=eq1.subs(x==x-x0,y==y-y0).expand()
    eq3=eq2.subs(x==x*cos(ang)+y*sin(ang), y==-x*sin(ang)+y*cos(ang)).expand()
    eq4=eq3.multiply_both_sides(eq3.left().denominator())
    return [x0,y0,ang,eq4]
parab_task=CanonToGeneralParab(parab)
G5=implicit_plot(parab_task[3].left(), (x,-50,50), (y,-50,50), axes='true', figsize=3, color = 'green')
# робим поверхности. сначала со всеми тремя переменными (12 вариантов)
ch1=randint(1,12)
ch2=randint(-1,1)
ch3=randint(1,2)
if ch3==1:
   ch3=-1
else:
   ch3=1
var('z')
abc=vector([randint(1,10) for i in range(3)])
aw=abc[0]
bw=abc[1]
cw=abc[2]
if ch1<=3:                                      
     eqconic=x^2/aw^2+y^2/bw^2+z^2/cw^2==1       
elif ch1==4:
     eqconic=x^2/aw^2+y^2/bw^2-z^2/cw^2==ch2
elif ch1==5:
     eqconic=x^2/aw^2-y^2/bw^2+z^2/cw^2==ch2
elif ch1==6:
     eqconic=-x^2/aw^2+y^2/bw^2+z^2/cw^2==ch2
elif ch1==7:
     eqconic=x^2/aw^2+y^2/bw^2==ch3*2*z
elif ch1==8:
     eqconic=x^2/aw^2+z^2/cw^2==ch3*2*y
elif ch1==9:
     eqconic=y^2/bw^2+z^2/cw^2==ch3*2*x
elif ch1==10:
     eqconic=x^2/aw^2-y^2/bw^2==ch3*2*z
elif ch1==11:
     eqconic=x^2/aw^2-z^2/cw^2==ch3*2*y
else:
     eqconic=y^2/bw^2-z^2/cw^2==ch3*2*x
x0y0z0=zero_vector(3)
while x0y0z0==zero_vector(3):
     x0y0z0=vector([randint(-10,10) for i in range(3)])
eq1=eqconic.subs(x==x-x0y0z0[0],y==y-x0y0z0[1],z==z-x0y0z0[2])
eq2=eq1.expand()
eq3=eq2.add_to_both_sides(-eq2.right())
eq4=eq3.multiply_both_sides(eq3.left().denominator())
conic_task=[x0y0z0, eq4]
# робим поверхности. теперь цилиндрические (12 вариантов)
ch1=randint(1,12)
ch2=randint(-1,1)
ch3=randint(1,2)
if ch3==1:
   ch3=-1
else:
   ch3=1
pabc=vector([randint(1,10) for i in range(4)])
aw=pabc[0]
bw=pabc[1]
cw=pabc[2]
pw=pabc[3]
if ch1==1:                                      
   eqcyl=x^2==ch3*2*pw*y       
elif ch1==2:
   eqcyl=y^2==ch3*2*pw*x
elif ch1==3:
   eqcyl=x^2==ch3*2*pw*z
elif ch1==4:
   eqcyl=z^2==ch3*2*pw*x
elif ch1==5:
   eqcyl=y^2==ch3*2*pw*z
elif ch1==6:
   eqcyl=z^2==ch3*2*pw*y
elif ch1==7:
   eqcyl=x^2/aw^2+y^2/bw^2==1
elif ch1==8:
   eqcyl=x^2/aw^2+z^2/cw^2==1
elif ch1==9:
   eqcyl=y^2/bw^2+z^2/cw^2==1
elif ch1==10:
   eqcyl=x^2/aw^2-y^2/bw^2==ch2
elif ch1==11:
   eqcyl=x^2/aw^2-z^2/cw^2==ch2
else:
   eqcyl=y^2/bw^2-z^2/cw^2==ch2
x0y0z0=zero_vector(3)
while x0y0z0==zero_vector(3):
   x0y0z0=vector([randint(-10,10) for i in range(3)])
eq1=eqcyl.subs(x==x-x0y0z0[0],y==y-x0y0z0[1],z==z-x0y0z0[2])
eq2=eq1.expand()
eq3=eq2.add_to_both_sides(-eq2.right())
eq4=eq3.multiply_both_sides(eq3.left().denominator())
cyl_task=[x0y0z0, eq4]
\end{sagesilent}

\vspace{2ex}	

\begin{question}{1}
	\begin{enumerate}
\item Составить каноническое уравнение \sagestr{txtcurve} с параметрами $a=\sage{a}$ и $b=\sage{b}$ в ДПСК, выписать координаты фокусов. Перейти к полярному уравнению и построить кривую в ПСК.
\item Составить каноническое уравнение параболы с вершиной в точке $(0, 0)$ с заданным фокальным параметром $p=\sage{p}$ и осью симметрии \textit{\sagestr{symm}}. Перейти к полярному уравнению и построить кривую в ПСК.
	\end{enumerate}
\end{question}
\begin{solution}
Уравнение \sagestr{txtcurve}: \ensuremath{\displaystyle{\sage{answ1}}}, фокусы $F_1\sage{F1}, F_2\sage{F2}$

Уравнение параболы: $\sage{answ2}$, фокус \ensuremath{\displaystyle{F\sage{F}}} 

Полярные уравнения: \ensuremath{\displaystyle{\sage{polar1}, \sage{polar2}}}

\sageplot{G1}\sageplot{G2}
\end{solution}

\begin{question}{1}
Для каждой точки кривой отношение расстояния до точки $A\sage{A}$ к расстоянию до прямой $l:\sage{l}$ равно \ensuremath{\displaystyle{\sage{n}}}. Найти уравнение кривой на плоскости, привести полученное уравнение к каноническому виду, указать тип кривой и, если это возможно, построить ее.	
\end{question}
\begin{solution}
\ensuremath{\displaystyle{\sage{answ3}}}
\end{solution}

\begin{question}{2}
Привести уравнение кривой второго порядка к каноническому виду и, если это возможно, построить ее.
\begin{enumerate}
	\item \ensuremath{\displaystyle{\sage{hyp_task[3]}}}
	\item \ensuremath{\displaystyle{\sage{elps_task[3]}}}
	\item \ensuremath{\displaystyle{\sage{parab_task[3]}}}
\end{enumerate}	
\end{question}
\begin{solution}
\begin{enumerate}
	\item \ensuremath{\displaystyle{\sage{hyperb}, \text{центр } (\sage{hyp_task[0]}, \sage{hyp_task[1]}), \angle\varphi=\sage{hyp_task[2]}}}
	\item \ensuremath{\displaystyle{\sage{elps}, \text{центр } (\sage{elps_task[0]}, \sage{elps_task[1]}), \angle\varphi=\sage{elps_task[2]}}}
    \item \ensuremath{\displaystyle{\sage{parab}, \text{центр } (\sage{parab_task[0]}, \sage{parab_task[1]}), \angle\varphi=\sage{parab_task[2]}}}
\end{enumerate}	

\sageplot{G3}\sageplot{G4}	

\sageplot{G5}
\end{solution}

\begin{question}{1}
Привести уравнение поверхности второго порядка к каноническому виду и построить ее.
\begin{enumerate}
	\item \ensuremath{\displaystyle{\sage{conic_task[1]}}}
    \item \ensuremath{\displaystyle{\sage{cyl_task[1]}}}
\end{enumerate}		
\end{question}
\begin{solution}
\begin{enumerate}
	\item \ensuremath{\displaystyle{\sage{eqconic}, \text{центр } \sage{conic_task[0]}}}
	\item \ensuremath{\displaystyle{\sage{eqcyl}, \text{центр } \sage{cyl_task[0]}}}
\end{enumerate}		
\end{solution}

\mycomment

\begin{tabular}{|l|*{\numberofquestions}{c|}c|}\hline
	Задача &
	\ForEachQuestion{\QuestionNumber{#1}\iflastquestion{}{&}} &
	Всего \\ \hline
	Баллы &
	\ForEachQuestion{\GetQuestionProperty{points}{#1}\iflastquestion{}{&}} &
	\pointssum* \\ \hline
	Набрано &
	\ForEachQuestion{\iflastquestion{}{&}} & \\ \hline
\end{tabular}

\end{document}
